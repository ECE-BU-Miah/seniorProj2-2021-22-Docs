\documentclass[letterpaper,12pt]{article}   %%% Can be report

%% set paper margins
\oddsidemargin=0.1in
\evensidemargin=0.1in
\textwidth=6.0in
\topmargin=-0.7in
\textheight=9.0in
\parindent=0.2in

\usepackage{amsmath,amssymb,bm}
\usepackage{graphicx}
\usepackage{rotating}
\usepackage{subfigure}
\graphicspath{%
  {figs/ipe/}
  {figs/dia/}
  {figs/matlab/}
  {figs/imag/}
} 
\usepackage[width=11cm,font=footnotesize,labelfont=bf, %
format=default,justification=centerlast]{caption} % Figure caption text customization 

\usepackage{pgfgantt}

\usepackage{booktabs,array} % Packages for tables

\usepackage{hyperref}
\usepackage{soul}
\usepackage{easyReview}
\usepackage{setspace}
\usepackage{multirow}

\usepackage[ruled, vlined, linesnumbered]{algorithm2e} % For algorithms

\usepackage{tikz}
\usetikzlibrary{positioning,fit}

\usepackage{siunitx}
\sisetup{unitsep=\cdot}

\title{ECE498:~Senior Capstone Project I\\\textbf{\underline{Project Proposal}}\\
\vspace{0.5in}
Project Title: Hardware-in-the-Loop Plant Modeling for Autonomous Vehicle 
\vspace{1.0in}
\author{Hannah Grady and Nicholas Nauman\\ Advisor: Dr.~Suruz Miah}
}
%\date{February 6, 2003}  No need to write, Date will be automatically on the title page
\date{}  % Do not show date on the title page

%%%%%%%%%%%%%%%%% Set document line spacing %%%%%%%%%%%%%%%%%%%%%
\singlespacing
%\onehalfspacing
%\doublespacing
% all packages are in the following tex file.
%\input{paper-preamble.tex}
\begin{document}

%%% Make title page 
\begin{titlepage}
 \maketitle

\vspace*{4.0cm}
\begin{center}
\normalsize
Electrical and Computer Engineering Department\\
Caterpillar College of Engineering and Technology\\
\href{http://www.bradley.edu/}{Bradley University}\\

\vspace*{6.0cm}
\copyright~H.~Grady~and~N.~Nauman, Peoria, IL, USA, 2021\\

\end{center}
\thispagestyle{empty}

\end{titlepage} 
%%%%%%%%%%%%
%\thispagestyle{empty}
%\maketitle
\newpage
\renewcommand{\contentsname}{Table of Contents}
\tableofcontents
\newpage

\section{Introduction}
Autonomous vehicles are being developed by many companies for commercial and/or personal use. These autonomous vehicles would allow companies to continue crucial deliveries or transports of their products even if there is a shortage of drivers. Also, autonomous vehicles have the ability to make roads safer for drivers and pedestrians alike. Accurate models of six main vehicle subsystems must be created so controllers can be developed to reliably manage these vehicles.
\vspace*{12pt}
\noindent
There are six main subsystems - a steering model, an acceleration model, a brake model, a shifting model, a speed model, and a cruise model - that must be modeled to get an accurate representation of how the vehicle should be controlled. We will be modeling these six subsystems for a Lexus vehicle platform using the System Identification Toolbox from Matlab. The first subsystem we will model is the steering model, and then we will move onto the other subsystems. There are already controllers in place for the Lexus vehicle platform, but their reliability is lower than expected due to non-linear behaviors of the torques voltages required to control each subsystem. The scope of this project includes developing mathematical models for each subsystem so AutonomouStuff can implement controls that improve the reliability of the autonomous vehicle platform. These models will be used by AutonomouStuff for their Hardware-in-the-Loop (HIL) testing.

 \section{Literature Review}
In the paper “Identification of Multiple Input-Single Output (MISO) Model for MPPT of Photovoltaic System”, the researchers wanted to model a system where the two inputs were Solar irradiance and Cell temperature and the output was DC current. In order to model this system Matlab’s System Identification Toolbox was used. In order to create different models, they collected and used data from an energy center in Malaysia. After generating different models, the researchers ended up going with a fourth order ARX (also known as ARXQS) model because it was the most accurate, with a best fit percentage of 93.42\%. The polynomial model equation for ARX is shown below. 

$y(t) + a_1y(t - 1) +...+a_{na}y(t - n_{a}) = b_{1}u(t-n_{k})+...+b_{nb}u(t - n_{k}-n_{b}+1) + e(t)$

$y(t)$ is the output of the equation at a certain time t, while u(t) is the input at a certain time t. The noise disturbance of the system is represented by e(t). The variables na, nb, and nk are the system’s number of poles, amount of b parameters, and the samples before the inputs begin to affect the system’s output.



\section{System Identification}

System identification is the process of developing mathematical models for a dynamic system using the measurement of input and output signals of that system. There are many components or systems that engineers use to accomplish a task that they are assigned without knowing the exact behavior of this system for a given input. Without knowing the response of this black-box, there could be unexpected consequences from a system. The goal of the system identification methodology is to get an accurate estimation of the system response to any given input. Mathworks' MATLAB has the system identification toolbox, where a few existing examples demonstrate the working principle of this toolbox.

\subsection{Example 1: Dealing with Multi-Variable Systems: Identification and Analysis}
\label{sec:sysID-Example1}
During this example I learned how to create an iddata object from a dataset in order to get the inputs and the outputs. The next step was to look at the impulse and step responses in order to learn more about how the inputs and outputs act. From there the state space model was estimated using the first part of the given data. This model was compared to the step responses that were found earlier and with the second part of the data to see if it was a good fit. The model had a best fit percentage of 83.55\% for the generated voltage data, and a best fit of 39.33\% for the speed data. The frequency response of the model was estimated with spectral analysis and bode plots were also created. The tutorial explained that if the data doesn’t give nice models, then it is best to try out submodels for the different channels. Two single-input single-output (SISO) models were created and compared with the existing multiple-input multiple-output (MIMO) model and the actual data. The Nyquist plots were also compared. Both SISO models performed well during these comparisons. The next step in the tutorial was to create a multiple-input single-output (MISO) model in order to get a model that more accurately reflected the generated voltage data. By creating this model and comparing it to the validation data and previous models, we saw a best fit percentage of 90.18\%. The last thing the example showed was how to merge the two SISO models we created earlier.


\subsection{Example 2: Selecting Model Structures for Multivariable Systems}
\label{sec:sysID-Example2}
This example discusses solutions for modeling both MISO and MIMO models using Mathworks' Matlab System Identification Toolbox. As the article discusses, MISO system models are easier to develop because all model structures used by the toolbox support models with a single output and multiple inputs. Therefore, the process for developing a model for a MISO system is importing the data as an iddata object, removing the mean from the data, and estimating the solution using any model structure available in the toolbox. The command line can be used by using the function associated with the model structure name and then using the compare function to get the best fit percentage. For MIMO systems, there are not model structures built into the System Identification Toolbox and they must be imported instead. Otherwise the process is very similar to that of a MISO system. For a MIMO system, using the compare function can be crucial. The compare function will tell which output channel is the most difficult to develop a model for, if it is possible at all. With this information, the output channel that is hardest to model should first be modeled individually because there will be less freedom in what model structures are available. The other channels should be able to closely relate to the model for the output channel you selected.


\subsection{Example 3: ..........}
\label{sec:sysID-Example3}



\section{System Requirements}
The vehicle platform plant model will meet the following requirements:
\begin{itemize}
    \item The plant model should have an accurate model of the six main subsystems
    \item A HIL test bench can be developed from the plant model of the subsystems
    \item The steering model can handle very small changes in steering angles
    \begin{itemize}
    		\item The steering model should have a very smooth, continuous output as a choppy, discontinuous output would translate into difficulty tracking the vehicle in a straight line for instances such as keeping the vehicle in its lane. For this reason, that model that we create should be able to smooth out any discontinuities that would normally be measured by the steering motor, especially for small changes in the steering angle
    \end{itemize}
\end{itemize}


\section{System Architecture}
The overall system architecture of this project consists of six subsystems which are the steering, acceleration, brake, speed, shift, and cruise control systems as shown in \autoref{fig:sys_block_diag}.

% \subsection{System Block Diagram}
% The high-level system block diagram of the proposed robotic cart (prototype) is shown in \autoref{fig:sys_block_diag}. There are four inputs to the proposed cart system. Obviously, the robotic cart is powered on through a source of power (input), which is simply a battery, for instance. There will be an on/off switch to allow the system to be powered down when not in use. The motion of the cart will be controlled by the motion of the remote. A stretch goal is to have mode selection buttons to allow the user to put the system into different operating modes which will be explained later on.

% \vspace*{12pt}
% \noindent
% The main output of the system is the trajectory of the cart on the ground. As the user moves with the remote, the cart is supposed to follow the user. The other output of the robotic cart system is status indication in the form of LED lights. These lights will be used to notify the user of the status of the system.

% \begin{figure}[!h]
%     \centering
%     % \includegraphics[scale=0.9]{figs/system_block_diagram_2}
%     \caption{System level block diagram detailing inputs and outputs to the
%       mobile cart system.}
% 	\label{fig:sys_block_diag}
% \end{figure}

% \subsection{Subsystem Block Diagrams}
% Both of the subsystems of the mobile cart system, being the Mobile Cart and Remote Target, act as their own enclosed systems. The two subsystems communicate with one another by relaying radio messages between them. The block diagram of the Remote Target subsystem is shown in \autoref{fig:remote_block_diag}. Of the two subsystems the Remote Target is the simplest since it only requires an XBee module attached to a 9 volt battery with a voltage regulation circuit.

% \vspace*{12pt}
% \noindent
% The Mobile Cart subsystem block diagram is shown in
% \autoref{fig:mobile_block_diag}. The Mobile Cart has four inputs. The cart
% requires a power source, which will be a Li-Po battery. The power to the
% subsystem will be toggled by an on/off switch. The buttons for changing the
% operating mode of the cart will be on the cart if we have time to implement
% them. The final input to the mobile cart subsystem is the XBee network messages
% received from the remote target.

% \vspace*{12pt}
% \noindent

% There are three outputs of the mobile cart subsystem. The first is the wheel
% velocities that move the cart. There will also be LEDs to indicate the status of
% the system to the user. Also, the cart will send radio messages to the remote
% target by means of an array of XBee radio modules. %
% %
% \begin{figure}[h!]
%   \centering
%   \missingfigure
%   % \includegraphics[scale=0.9]{figs/remote_target_block_diagram}
%   \caption{Remote Target block diagram}
%   \label{fig:remote_block_diag}
% \end{figure}

% \begin{figure}[h!]
%   \centering
%   % \includegraphics[scale=0.82]{figs/mobile_cart_block_diagram}
%   \caption{Block diagram showing the subsystem-level components of the proposed robotic cart.}
%   \label{fig:mobile_block_diag}
% \end{figure}


% \subsection{Operation of Mobile Cart System}
% The Mobile Cart will controlled by a central embedded computer which we are using the BeagleBone Blue for. Two DC motors will be used to drive wheels to move the cart. Five XBee modules will be used to allow communication with the remote target. One of these radio sensors will be mounted on top of the cart to broadcast in all directions. The other four sensors will be placed inside parabolic reflectors at right angles to each other. This sensor array will be mounted on a stepper motor to allow rotation.

% \vspace*{12pt}
% \noindent

% There are two main steps in the operation of the robotic cart system. The first step is localization of the remote in the robot's local coordinate frame. Once the location of the remote is known, the navigation step handles the control of the wheel velocities to move the cart to the desired location. A flowchart of the system operation is shown in \autoref{fig:sysOpFlowchart}.

% \begin{figure}
%   \centering
%   \input{figs/dia/systemOperationFlowchart.tex}
%   \caption{Flowchart of System Operation}
%   \label{fig:sysOpFlowchart}
% \end{figure}


 \subsection{Specifications}
 Based on the system requirements, the vehicle platform plant model must meet the following specifications:
 \begin{itemize}
    \item The model of the steering subsystem should be modeled first as it is crucial to operation and has non-linearity considerations
    \item All subsystems will be modeled individually 
    \item Our steering model will be designed to accommodate very small steering angles
 \end{itemize}

\section{Preliminary Work}
To start this project, we first read documentation outlining the uses of Matlab's System Identification Toolbox. From there we worked on Matlab tutorials on how to model systems from data using System Identification and then find the most accurate model. We specifically tried to find examples with Multiple-Input Multiple-Output (MIMO) and Multiple-Input Single-Output (MISO) systems, since most of the vehicle subsystems we will model fall into one of these categories. A literature review was also conducted to see how systems with small non-linearities like our steering subsystem were modeled using System Identification. On October 7th we traveled to AutonomouStuff in order to collect data from the steering, acceleration, and braking subsystems in an autonomous vehicle. The data we collected will be used to generate and then verify our models. 

% % \subsection{Modelling} \label{sec:model}

% % \subsection{Simulation Results} \label{sec:simresults}

% % \subsection{Design} \label{sec:design}

% % \subsection{Experimental Activities} \label{sec:expresults}


\section{Parts List}
	\subsection{Software}
	\begin{itemize}
    \item MATLAB's System Identification Toolbox 
    \item  
 \end{itemize} 
	
	\subsection{Hardware} 
	\begin{itemize}
    \item Laptop
    \item PAC Mod 
    \item CAN Case 
    \item CAN Bus 
 \end{itemize}

\section{Timeline and Milestones} \label{sec:timeline}

 \begin{figure}
   \centering
   \begin{ganttchart}[
     hgrid,
     vgrid,
     x unit=.6cm,
     y unit title=.8cm,
     y unit chart=.6cm,
     milestone label font=\tiny,
     milestone progress label font = \tiny,
     milestone progress label anchor = east,
     bar label font=\tiny,
     group label font=\small,
     bar/.append style={fill=green},
     bar incomplete/.append style={fill=red},
     group progress label font = \tiny,
     progress label text={$\displaystyle#1\%$},
     group progress label anchor = east,
     bar progress label font = \tiny,
     bar progress label anchor = east,
     ]{1}{17}

%     \gantttitle{2020}{17}\\
     \gantttitle{Sep}{4}
     \gantttitle{Oct}{4}
     \gantttitle{Nov}{4}
     \gantttitle{Dec}{4}
     \gantttitle{}{1}\\

%     \ganttgroup[progress = 100]{Research}{3}{12} \\
     \ganttbar[progress = 100]{Read System Identification Documentation}{3}{12}\\
     \ganttbar[progress = 100]{Collect Steering, Acceleration, and Braking Data}{3}{12}\\

%     \ganttgroup[progress = 80]{Simulation}{8}{10}\\
     \ganttbar[progress = 80]{System Identification Tutorials}{8}{10}\\
     \ganttbar[progress = 100]{System Functional Requirements Document}{8}{10}\\
          \ganttbar[progress = 100]{System Functional Requirements Presentation}{8}{10}\\

     \ganttbar[progress = 100]{Parts Order Request}{8}{12}\\
     \ganttbar[progress = 100]{Proposal Document}{11}{11}\\
     \ganttbar[progress = 100]{Proposal Presentation}{12}{12}\\

%     \ganttgroup[progress = 82.5]{XBee Setup}{13}{14}\\
     \ganttmilestone[progress = 90]{Model Steering System}{13}{13}\\
     \ganttbar[progress = 80]{Model Acceleration System}{14}{14}\\
     \ganttbar[progress = 90]{Model Braking System}{14}{14}\\
     
   \end{ganttchart}
 \caption{Gantt chart for Fall 2021}
 \label{fig:gantt1}
 \end{figure}

 \begin{figure}
   \centering
   \begin{ganttchart}[
     hgrid,
     vgrid,
     x unit=.6cm,
     y unit title=.8cm,
     y unit chart=.6cm,
     milestone label font=\tiny,
     milestone progress label font = \tiny,
     milestone progress label anchor = east,
     bar label font=\tiny,
     group label font=\small,
     bar/.append style={fill=green},
     bar incomplete/.append style={fill=red},
     group progress label font = \tiny,
     progress label text={$\displaystyle#1\%$},
     group progress label anchor = east,
     bar progress label font = \tiny,
     bar progress label anchor = east,
     ]{1}{18}
     %\gantttitle{2021}{18}\\
     \gantttitle{Jan}{2}
     \gantttitle{Feb}{4}
     \gantttitle{Mar}{4}
     \gantttitle{Apr}{4}
     \gantttitle{May}{4}\\

%     \ganttgroup[progress = 0]{Assembly}{3}{4}\\
     \ganttbar[progress = 0]{Collect Shift, Speed Control, and Speed Data}{3}{3}\\
     \ganttbar[progress = 0]{Model Shift System}{3}{4}\\
     \ganttbar[progress = 0]{Model Speed Control System}{4}{4}\\
     \ganttbar[progress = 0]{Model Speed System}{4}{10}\\

%     \ganttgroup[progress = 10]{Software}{5}{6}\\
     \ganttbar[progress = 20]{Test Subsystems}{5}{5}\\
     \ganttmilestone[progress = 0]{Test Subsystems with HIL}{6}{6}\\

%     \ganttgroup[progress = 0]{Project Completion}{16}{17}\\
     \ganttmilestone[progress = 0]{Final Report}{16}{16}\\
     \ganttmilestone[progress = 0]{Final Presentation}{17}{17}\\
     \ganttbar[progress = 0]{Presentation to IAB}{17}{17}\\
     \ganttbar[progress = 0]{Project Demo}{17}{17}\\

%     \ganttmilestone[progress = 0]{Project Complete}{17}
   \end{ganttchart}
   \caption{Gantt Chart for Spring 2022}
   \label{fig:gantt2}
 \end{figure}

% The Gantt charts in Figures \ref{fig:gantt1} and \ref{fig:gantt2} show our planned schedule to complete this project. In these charts, there are four sections in each month represented by the dotted grid. This is to approximate a weekly schedule. Our first milestone is to complete the simulation of the system in MATLAB and CoppeliaSim. This includes simulating a remote as well as the robotic cart. The simulation of the RSSI detection will be handled in Matlab based on the positions of the cart and remote. We will add noise to the simulated RSSI in order to simulate the multipath effect. The expected functionality is that the cart will follow the remote.

% The second milestone is to complete the assembly of both the cart and the remote. The cart assembly involves replacing the existing motors with the purchased motors, as well as mounting the stepper motor, reflector, and XBee on top of the robot. The assembly of the remote involves constructing the voltage regulator circuit.

% The third milestone is to integrate the subsystems into one system. This will involve testing the angle and distance estimation from the cart to the remote. It will also include testing the following capabilities of the overall system. We expect that this will require a large amount of debugging and tuning. Therefore, we have allotted several weeks for this purpose.

% The fourth and final milestone is to complete the final report and presentation. This involves documenting our findings in a report and presenting our work. At this point, the project will be complete.

\section{Concluding Remarks}
Autonomous vehicles have a numerous amount of useful applications in today's society. From being used for delivering packages to allowing drivers to rest while traveling, as the technology for autonomous vehicles is enhanced, they will become an enticing product that many consumers will purchase and use. One of the biggest concerns with autonomous vehicles today is the risk associated with public safety. One solution to reducing the safety risk is ensuring that accurate controllers are developed to reliably control each subsystem involved with the controlling the movement of the autonomous vehicle. Due to non-linear behaviors, there is no simple solution for developing reliable controllers. This project will focus on deriving models of some main subsystems so that these non-linear behaviors are removed and smooth controllers can be developed.

\pagebreak
\bibliographystyle{IEEEtran}
\bibliography{bib/references.bib}

\end{document} 

%%% Local Variables:
%%% mode: latex
%%% TeX-master: t
%%% End:
