%%%%%%%%%%%%%%%%%%%%%%%%%%%%%%%%%%%%%%%%%
% Jacobs Landscape Poster
% LaTeX Template
% Version 1.0 (29/03/13)
%
% Created by:
% Computational Physics and Biophysics Group, Jacobs University
% https://teamwork.jacobs-university.de:8443/confluence/display/CoPandBiG/LaTeX+Poster
% 
% Further modified by:
% Nathaniel Johnston (nathaniel@njohnston.ca)
%
% This template has been downloaded from:
% http://www.LaTeXTemplates.com
%
% License:
% CC BY-NC-SA 3.0 (http://creativecommons.org/licenses/by-nc-sa/3.0/)
%
%%%%%%%%%%%%%%%%%%%%%%%%%%%%%%%%%%%%%%%%%

%----------------------------------------------------------------------------------------
%	PACKAGES AND OTHER DOCUMENT CONFIGURATIONS
%----------------------------------------------------------------------------------------

\documentclass[final]{beamer}

\usepackage[scale=1.24]{beamerposter} % Use the beamerposter package for laying out the poster

\usepackage{tikz}
\usepackage{todonotes}
\usepackage{epstopdf}
\epstopdfsetup{suffix = {}}
\usepackage{hyperref}
\usepackage{subfigure}  % Written by Steven Douglas Cochran
\usepackage{amssymb,amsmath,bm}    % From the American Mathematical Society
\usepackage{siunitx} % for units like degree, ...
\sisetup{unitsep=\cdot,binary-units=true}
\usepackage{mathtools}
\usepackage{xcolor,soul}
\usepackage{geometry}
\usetheme{confposter} % Use the confposter theme supplied with this template

\setbeamercolor{block title}{fg=BUred,bg=white} % Colors of the block titles
\setbeamercolor{block body}{fg=black,bg=white} % Colors of the body of blocks
\setbeamercolor{block alerted title}{fg=BUred,bg=gray!70} % Colors of the highlighted block titles
\setbeamercolor{block alerted body}{fg=black,bg=gray!10} % Colors of the body of highlighted blocks


% Many more colors are available for use in beamerthemeconfposter.sty

%-----------------------------------------------------------
% Define the column widths and overall poster size
% To set effective sepwid, onecolwid and twocolwid values, first choose how many columns you want and how much separation you want between columns
% In this template, the separation width chosen is 0.024 of the paper width and a 4-column layout
% onecolwid should therefore be (1-(# of columns+1)*sepwid)/# of columns e.g. (1-(4+1)*0.024)/4 = 0.22
% Set twocolwid to be (2*onecolwid)+sepwid = 0.464
% Set threecolwid to be (3*onecolwid)+2*sepwid = 0.708

\newlength{\sepwid}
\newlength{\onecolwid}
\newlength{\twocolwid}
\newlength{\threecolwid}
\setlength{\paperwidth}{48in} % A0 width: 46.8in
\setlength{\paperheight}{36in} % A0 height: 33.1in
\setlength{\sepwid}{0.022\paperwidth} % Separation width (white space) between columns
\setlength{\onecolwid}{0.22\paperwidth} % Width of one column
\setlength{\twocolwid}{0.464\paperwidth} % Width of two columns
\setlength{\threecolwid}{0.708\paperwidth} % Width of three columns
\setlength{\topmargin}{-0.25in} % Reduce the top margin size
\setlength{\leftmargin}{-0.25in}
\setlength{\rightmargin}{-0.25in}

%-----------------------------------------------------------

\usepackage{graphicx}  % Required for including images

\usepackage{booktabs} % Top and bottom rules for tables
%\usepackage{subfigure}

%-----------------------------------------------------------
% TITLE SECTION
%-----------------------------------------------------------

\title{Plant Modeling for an Autonomous Vehicle} % Poster title

\author{Nick Nauman, Hannah Grady, Advisor: Dr. Suruz Miah} % Author(s)


\institute{Department of Electrical and Computer Engineering, Bradley University, Peoria IL} %Institution(s)
\vskip -.5cm

%-----------------------------------------------------------
% POSTER CONTENT
%-----------------------------------------------------------

\begin{document}

\addtobeamertemplate{block end}{}{\vspace*{1ex}} % White space under blocks
\addtobeamertemplate{block alerted end}{}{\vspace*{2ex}} % White space under highlighted (alert) blocks

\setlength{\belowcaptionskip}{2ex} % White space under figures
%\setlength\belowdisplayshortskip{2ex} % White space under equations

\begin{frame}[t] % The whole poster is enclosed in one beamer frame

\begin{columns}[t]

\begin{column}{\sepwid}\end{column} % Empty spacer column

\begin{column}{\onecolwid} % The first column

%-----------------------------------------------------------
% OBJECTIVE AND CONTRIBUTION
%-----------------------------------------------------------

\begin{alertblock}{Objective and Contribution}
%
\textbf{Objective}
\vskip -0.75cm
\begin{itemize}
    \item Create models for the subsystems of an autonomous vehicle
\end{itemize}
%\begin{itemize}
%    \item Develop a platform that allows mobile devices to control helicopters with the least amount of error via an embedded system 
%\end{itemize}
%
\vskip -1cm
\textbf{Contribution}
\vskip -0.75cm
\begin{itemize}
	\item Determine if System Identification or Neural Network modeling produces better models 
	\item Non-linearity modeling
%    \item Determine trade-offs between traditional control techniques and machine learning
    %\item Embedded System Implementation
    %\item Mobile Device Control 
    %\item Multi-Helicopter Application
\end{itemize}
\vskip -0.75cm
\textbf{Applications}
\vskip -0.75cm
\begin{itemize}
	\item Use in testing to help develop more accurate vehicle controllers
	\item Create a guide for modeling future vehicle subsystems
%    \item Teleoperation approach to search and rescue 
%    \item Aerial turbulence resistance 
\end{itemize}

%


\end{alertblock}

%-----------------------------------------------------------
% PROBLEM SETUP
%-----------------------------------------------------------
\vskip -1cm
\begin{block}{Problem Setup}
\vskip -1cm
%\begin{enumerate}
%    \item Develop and linearize system model
%    \item Simulate algorithms in MATLAB
%    \item Test algorithms on desktop computer using USB connection 
%    \item Test algorithms on Raspberry Pi microcontroller
%    \item Test algorithms on mobile device
%\end{enumerate}

\begin{figure}
    \centering
    % \fcolorbox{gray!10}{gray!10}{\includegraphics[height=.15\textheight,width=\textwidth,keepaspectratio=true]{figs/ProblemStatementImage_Gray2}}
    % \fcolorbox{gray!20}{gray!20}{\includegraphics[height=.15\textheight,width=\textwidth,keepaspectratio=true]{figs/ipe/highLevel_gry_grn.eps}
    % }
    \caption{High level architecture of the proposed system.}
    \label{fig:highLevelDiagram}
\end{figure}
\begin{figure}
    \centering
    % \missingfigure{Insert figure.}
    % \fcolorbox{gray!10}{gray!10}{
    % \includegraphics[height=.2\textheight,width=.92\textwidth,keepaspectratio=true]{figs/img/quanserAero.png}
    % }
    \caption{2-DOF helicopter (Quanser Aero).}
    %\caption{DC motors are attached to main and tail rotors which allow the helicopter to change its pitch and yaw respectively.}
    \label{fig:helicoterModel}
\end{figure}
\vskip -1cm
$\bullet$ State-space representation of 2-DOF helicopter
\begin{align*}
\begin{bmatrix}
    \dot\theta\\
    \dot\psi\\
    \ddot{\theta}\\
    \ddot{\psi}
\end{bmatrix}&=
%\label{eq:matrixA}
\begin{bmatrix}
    0 & 0 & 1 & 0 \\
    0 & 0 & 0 & 1 \\
    0 & -K_{sp}/J_p & -D_p/J_p & 0 \\
    0 & 0 & 1 & -D_y/J_y 
\end{bmatrix} 
%\label{eq:stateMatrix} 
\begin{bmatrix}
    \theta\\
    \psi\\
    \dot{\theta}\\
    \dot{\psi}
\end{bmatrix}
\\&+
%\label{eq:matrixB}
\begin{bmatrix}
    0 & 0 \\
    0 & 0 \\
    K_{pp}/J_p & K_{py}/J_p \\
    K_{yp}/J_y & K_{yy}/J_y 
\end{bmatrix}
%\label{eq:inputMatrix}
\begin{bmatrix}
    V_p \\
    V_y 
\end{bmatrix}
\end{align*}

\vskip -1cm
\end{block}

\end{column} % End of the first column

\begin{column}{\sepwid}\end{column} % Empty spacer column

\begin{column}{\onecolwid} % The second column

%-----------------------------------------------------------
% LQR
%-----------------------------------------------------------

\begin{block}{Motion (Trajectory) Control Algorithm}
\vskip -1cm
%\begin{itemize}
%    \item Determines optimal control gain for dynamic systems
%%    \begin{subequations}
%%        \small
%%        \begin{align}
%%            & \dot{x} = Ax + Bu\\
%%            & u = -Kx\\
%%            & J(u) = \int_{0}^{\infty} (x^TQx + u^TRu + 2x^TNu)dt\\
%%            & A^TS+SA-(SB+N)R^{-1}(B^TS+N^T)+Q=0\\
%%            & K=R^{-1}(B^TS+N^T)
%%        \end{align}
%%    \label{eq:LQR_EQ}
%%    \end{subequations}
\begin{figure}
    \centering
    % \missingfigure{Insert figure.}
    % \fcolorbox{gray!10}{gray!10}{\includegraphics[height=.15\textheight,width=.92\textwidth,keepaspectratio=true]{figs/ipe/algorithmFramework}}
    \caption{A desired orientation is given by a user.  The difference between this input and the actual position is calculated.  The controller the calculates the proper amount of voltage to apply to the DC motors.}
    \label{fig:algorithmFramework}
\end{figure}
%\end{itemize} 
\vskip -1cm
\begin{enumerate}
    \item Employ state-space representation of 2-DOF helicopter:
    \begin{align*}
        \dot{\mathbf{x}} = \mathbf{A}\mathbf{x} + \mathbf{B}\mathbf{u}
    \end{align*}
    \item Use state feedback law
    \begin{center}
        \vskip -1cm
        $\mathbf{u} = -\mathbf{K}\mathbf{x}$
    \end{center}
    \vskip -1cm
    to minimize the quadratic cost function:
    \begin{align*}
        J(\mathbf{u}) = \int_0^\infty (\mathbf{x}^T\mathbf{Q}\mathbf{x} + \mathbf{u}^T\mathbf{R}\mathbf{u} + 2\mathbf{x}^T\mathbf{N}\mathbf{u})\mathrm{dt}
    \end{align*}
    \item Find the solution $\mathbf{S}$ to the Riccati equation
    \begin{align*}
        \mathbf{A}^T\mathbf{S}+\mathbf{SA}-(\mathbf{SB}+\mathbf{N})\mathbf{R}^{-1}(\mathbf{B}^T\mathbf{S}+\mathbf{N}^T)+\mathbf{Q}=0
    \end{align*}    
    \item Calculate gain, $\mathbf{K}$
    \begin{center}
        $\mathbf{K}=\mathbf{R}^{-1}(\mathbf{B}^T\mathbf{S}+\mathbf{N}^T)$
    \end{center}
\end{enumerate}

%\vskip -2.5cm
\end{block}

%-----------------------------------------------------------
% LQG
%-----------------------------------------------------------

\begin{block}{Optimal Noise Resistant Control Algorithm}
\vskip -1cm
\begin{itemize}
    \item Utilizes gain calculated in LQR
    \item Added Kalman filter to reduce external disturbances to the system
%    \begin{subequations}
%        \small
%        \begin{align}
%            & some eq = x\\
%            & some eq = y\\
%            & some eq = z
%        \end{align}
%    \label{eq:LQG_EQ}
%    \end{subequations}
\end{itemize} 
%\todo[inline]{Insert Dr. Miah's LQG image from his slides here}
\begin{figure}
    \centering
    % \missingfigure{Insert figure.}
    % \fcolorbox{gray!10}{gray!10}{\includegraphics[width=1.05\textwidth,keepaspectratio=true]{figs/img/LQG_SimulinkResize.png}}
    \caption{Noise resistant 2-DOF helicopter model.}
    \label{fig:LQGModel}
\end{figure}

%\vskip -2.5cm
\end{block}



\end{column} % End of second column

\begin{column}{\sepwid}\end{column} % Empty spacer column

\begin{column}{\onecolwid} % The third column
%-----------------------------------------------------------
% ADP
%-----------------------------------------------------------

\begin{block}{Reinforcement Learning Algorithm}
\vskip -1cm
\begin{itemize}
    %\item Reinforcement learning based approach
    \item Uses neural network based on difference between desired and actual orientation to determine optimal gain
    %\item Unlike LQR, ADP operates only in discrete-time and calculates new gains as the system is running
%    \begin{subequations}
%        \small
%        \begin{align}
%            & some eq = x\\
%            & some eq = y\\
%            & some eq = z
%        \end{align}
%    \label{eq:ADP_EQ}
%    \end{subequations}
\end{itemize} 
%\begin{enumerate}
%    \item Apply input to the system and calculate error e(k) at time k
%    \item Use Euler's method to convert the error model into discrete-time and record data every $\tau \sec$
%    \begin{center}
%        $e_{k+1}=f(e_k)+G(e_k)u_k$
%    \end{center}
%    \item 
%    %\item Minimize the cost function
%    %\begin{center}
%        $J(u)=\sum_{k=0}^{\infty} e_k^TQe_k+u_k^TRu_k$
%    %\end{center}
%    %\item Define value function
%    %\begin{center}
%        $V(e_k)=$
%    %\end{center}    
%    \item Solve for gain, K
%    \begin{center}
%        $K=0.5R^{-1}B^TP$
%    \end{center}
%\end{enumerate}

\begin{figure}
    \centering
    % \fcolorbox{gray!10}{gray!10}{\includegraphics[width=.9\textwidth,keepaspectratio=true]{figs/ipe/ADP_Neural_Network.eps}}
    \caption{ADP Neural Network}
    \label{fig:NeuralNetwork}
\end{figure}



\vskip 1.5cm
%(1) t = kτ where τ is given
%(2)Advance the system by applying determined inputs to the system and measure system states
%(3)Calculate state error as e[k] = xref[k] − x[k], save state error
%(4) If t 6= T, increment k and start at (1)
%(5) If t = T, update K using collected error data
%(5a)Use error states since last update of K as inputs to the neural network
%(5b)Recursively determine wc values for each of the instances
%(5c)Approximate new wc value using regression analysis of the individually calculated wc values
%(6)Calculate new K using wc values
%(7) Increment k and start at (1)


\vskip -2.5cm
\end{block}
%-----------------------------------------------------------
% Subsystem Block Diagram
%-----------------------------------------------------------
%\begin{block}{Subsystem Block Diagram}
%\begin{figure}
%    \centering
%        \missingfigure{Insert figure.}
%    % \boxed{\includegraphics[width=.92\textwidth,keepaspectratio=true]{figs/ipe/subsystemBlockDiagramAlternate2}}
%    \label{fig:subsystem}
%\end{figure}
%\end{block}

%-----------------------------------------------------------
% SIMULATION RESULTS
%-----------------------------------------------------------

\begin{block}{Simulation Results}
\vskip -1cm
%\begin{itemize}
%    \item \todo[inline]{add LQR and LQG simulation results}
%\end{itemize}

%Before implementing the algorithm, it was simulated using the industry standard robot simulator V-REP. With numerous robot models available, V-REP is able to accurately and consistently produce like those expected in real--world scenarios. Preliminary simulation results can be seen in Fig.~\ref{fig:vrepResults}.

\begin{figure}
    \centering
    \subfigure[][]{
    %\missingfigure{Insert figure.}
    % \includegraphics[width=.46\textwidth,keepaspectratio=true]{figs/matlab/LQG_PIvLQR_PI_Sim/Pitch}
    \label{fig:simStepPitch}
    }
    \subfigure[][]{
    %\missingfigure{Insert figure.}
    % \includegraphics[width=.46\textwidth,keepaspectratio=true]{figs/matlab/LQG_PIvLQR_PI_Sim/Yaw}
    \label{fig:simStepYaw}
    }
    \subfigure[][]{
    %\missingfigure{Insert figure.}
    % \includegraphics[width=.46\textwidth,keepaspectratio=true]{figs/matlab/LQG_PIvLQR_PI_Sim/Pitch_Volt}
    \label{fig:simPitchVolt}
    }
    \subfigure[][]{
    %\missingfigure{Insert figure.}
    % \includegraphics[width=.46\textwidth,keepaspectratio=true]{figs/matlab/LQG_PIvLQR_PI_Sim/Yaw_Volt}
    \label{fig:simYawVolt}
    }
    \caption{A comparison between LQG and LQR control for a step input is shown for \subref{fig:simStepPitch}~the~main~rotor and \subref{fig:simStepYaw}~the~tail~rotor and the corresponding voltages in \subref{fig:simPitchVolt}~and~\subref{fig:simYawVolt}}
    \label{fig:simResults}
\end{figure}
\vskip -2cm
\end{block}


\end{column} % End of third column

\begin{column}{\sepwid}\end{column}

\begin{column}{\onecolwid} % The fourth column

%-----------------------------------------------------------
% EXPERIMENTAL RESULTS
%-----------------------------------------------------------

\begin{block}{Experimental Results}
\vskip -1cm
%\begin{itemize}
%    \item ADP Algorithm
%\end{itemize}
\begin{figure}
    \centering
    % \fcolorbox{gray!10}{gray!10}{\includegraphics[height=.15\textheight,width=.9\textwidth,keepaspectratio=true]{figs/ipe/setup.eps}}
    \caption{Experimental Setup}
    \label{fig:Setup}
\end{figure}

\begin{figure}
    \centering
    \subfigure[][]{
    %\missingfigure{Insert figure.}
    % \includegraphics[width=.46\textwidth,keepaspectratio=true]{figs/matlab/ADPvLQR_P_USB/Pitch_ADP_LQR}
    \label{fig:stepPitchADP}
    }
    \subfigure[][]{
    %\missingfigure{Insert figure.}
    % \includegraphics[width=.46\textwidth,keepaspectratio=true]{figs/matlab/ADPvLQR_P_USB/Yaw_ADP_LQR}
    \label{fig:stepYawADP}
    }
    \caption{ADP experimental results for \subref{fig:stepPitchADP}~the~main~rotor and \subref{fig:stepYawADP}~the~tail~rotor given a step input}
    \label{fig:ADP_Results}
\end{figure}
%\begin{itemize}
%    \item Comparison between different controller architectures (proportional controller and proportional controller with integral term)
%\end{itemize}
\vskip -.5cm

\begin{figure}
    \centering
    \subfigure[][]{
    % \includegraphics[width=.46\textwidth,keepaspectratio=true]{figs/matlab/LQR_PIvLQR_P_USB/step/Pitch_LQR_RMSE}
    \label{fig:stepPitch}
    }
    \subfigure[][]{
    % \includegraphics[width=.46\textwidth,keepaspectratio=true]{figs/matlab/LQR_PIvLQR_P_USB/step/Yaw_LQR_RMSE}
    \label{fig:stepYaw}
    }
    \caption{Comparison between P and PI control for a step input is shown for \subref{fig:stepPitch}~the~main~rotor and \subref{fig:stepYaw}~the~tail~rotor}
    \label{fig:expResults}
\end{figure}
\vskip -2cm
\end{block}

\begin{figure}
  \centering
  \subfigure[][]{
  \label{fig:time0}
  % \includegraphics[width=0.4\textwidth]{figs/img/timeEquals0.jpg}
  }
  \subfigure[][]{
  \label{fig:time10}
  % \includegraphics[width=0.4\textwidth,height=0.057\textheight]{figs/img/timeEquals10_2}
  }  
  \caption{\subref{fig:time0} Time = 0~and~\subref{fig:time10} Time = 10}
  \label{fig:posistionAtTime}
\end{figure}

%\begin{block}{Conclusion}

%As seen in the experimental results, this system brings moderate accuracy considering its implementation. With a total cost of around \$140 for the system mounted on the mobile robot and around \$23 for the beacons, this system is much more attainable for those conducting research in mapping and localization algorithms and inexpensive to scale to commercial applications.

%\end{block}

%-----------------------------------------------------------
% Conclusion and Future Work
%-----------------------------------------------------------

\begin{alertblock}{Conclusion and Future Work}
\begin{itemize}
    \item Using Neural Networks produced more accurate models than System Identification 
   % \item PI controller greatly reduces steady-state error, however increases overshoot and settling time
    \vskip .75cm
    \item Test models using Hardware-in-the-Loop
    \item Create new vehicle controllers
\end{itemize}


\end{alertblock}

\end{column} % End of the fourth column

\end{columns} % End of all the columns in the poster

\end{frame} % End of the enclosing frame

\end{document}

%%% Local Variables:
%%% mode: latex
%%% TeX-master: t
%%% End:
