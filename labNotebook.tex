%%%%%%%%%%%%%%%%%%%%%%%%%%%%%%%%%%%%%%%%
% Compact Laboratory Book
% LaTeX Template
% Version 1.0 (4/6/12)
%
% This template has been downloaded from:
% http://www.LaTeXTemplates.com
%
% Original author:
% Joan Queralt Gil (http://phobos.xtec.cat/jqueralt) using the labbook class by
% Frank Kuster (http://www.ctan.org/tex-archive/macros/latex/contrib/labbook/)
%
% License:
% CC BY-NC-SA 3.0 (http://creativecommons.org/licenses/by-nc-sa/3.0/)
%
% Important note:
% This template requires the labbook.cls file to be in the same directory as the
% .tex file. The labbook.cls file provides the necessary structure to create the
% lab book.
%
% The \lipsum[#] commands throughout this template generate dummy text
% to fill the template out. These commands should all be removed when 
% writing lab book content.
%
% HOW TO USE THIS TEMPLATE 
% Each day in the lab consists of three main things:
%
% 1. LABDAY: The first thing to put is the \labday{} command with a date in 
% curly brackets, this will make a new section showing that you are working
% on a new day.
%
% 2. EXPERIMENT/SUBEXPERIMENT: Next you need to specify what 
% experiment(s) and subexperiment(s) you are working on with a 
% \experiment{} and \subexperiment{} commands with the experiment 
% shorthand in the curly brackets. The experiment shorthand is defined in the 
% 'DEFINITION OF EXPERIMENTS' section below, this means you can 
% say \experiment{pcr} and the actual text written to the PDF will be what 
% you set the 'pcr' experiment to be. If the experiment is a one off, you can 
% just write it in the bracket without creating a shorthand. Note: if you don't 
% want to have an experiment, just leave this out and it won't be printed.
%
% 3. CONTENT: Following the experiment is the content, i.e. what progress 
% you made on the experiment that day.
%
%%%%%%%%%%%%%%%%%%%%%%%%%%%%%%%%%%%%%%%%%

%----------------------------------------------------------------------------------------
%    PACKAGES AND OTHER DOCUMENT CONFIGURATIONS
%----------------------------------------------------------------------------------------                               

% \UseRawInputEncoding

\documentclass[fontsize=11pt, % Document font size
                             paper=letter, % Document paper type
                             %twoside, % Shifts odd pages to the left for easier reading when printed, can be changed to oneside
                             openany, % chapters can start on any page
                             captions=tableheading,
                             index=totoc,
                             hyperref]{labbook}

%\documentclass[idxtotoc,hyperref,openany]{labbook} % 'openany' here removes the
  
\usepackage[bottom=10em]{geometry} % Reduces the whitespace at the bottom of the page so more text can fit

\usepackage[english]{babel} % English language
\usepackage{lipsum} % Used for inserting dummy 'Lorem ipsum' text into the template

\usepackage[utf8]{inputenc} % Uses the utf8 input encoding
\usepackage[T1]{fontenc} % Use 8-bit encoding that has 256 glyphs

\usepackage[osf]{mathpazo} % Palatino as the main font
\linespread{1.05}\selectfont % Palatino needs some extra spacing, here 5% extra
\usepackage[scaled=.88]{beramono} % Bera-Monospace
\usepackage[scaled=.86]{berasans} % Bera Sans-Serif

\usepackage{booktabs,array} % Packages for tables

\usepackage{amsmath} % For typesetting math
\usepackage{graphicx} % Required for including images
\usepackage{etoolbox}
\usepackage[norule]{footmisc} % Removes the horizontal rule from footnotes
\usepackage{lastpage} % Counts the number of pages of the document
\usepackage{float}

\usepackage[ruled, vlined, linesnumbered]{algorithm2e} % For algorithms


\usepackage[dvipsnames]{xcolor}  % Allows the definition of hex colors
\usepackage{epstopdf}
\epstopdfsetup{suffix={}}
\definecolor{titleblue}{rgb}{0.16,0.24,0.64} % Custom color for the title on the title page
\definecolor{linkcolor}{rgb}{0,0,0.42} % Custom color for links - dark blue at the moment

\addtokomafont{title}{\Huge\color{titleblue}} % Titles in custom blue color
\addtokomafont{chapter}{\color{OliveGreen}} % Lab dates in olive green
\addtokomafont{section}{\color{Sepia}} % Sections in sepia
\addtokomafont{pagehead}{\normalfont\sffamily\color{gray}} % Header text in gray and sans serif
\addtokomafont{caption}{\footnotesize\itshape} % Small italic font size for captions
\addtokomafont{captionlabel}{\upshape\bfseries} % Bold for caption labels
\addtokomafont{descriptionlabel}{\rmfamily}
\setcapwidth[c]{10cm} % Center align caption text
\setkomafont{footnote}{\sffamily} % Footnotes in sans serif

\deffootnote[4cm]{4cm}{1em}{\textsuperscript{\thefootnotemark}} % Indent footnotes to line up with text

\DeclareFixedFont{\textcap}{T1}{phv}{bx}{n}{1.5cm} % Font for main title: Helvetica 1.5 cm
\DeclareFixedFont{\textaut}{T1}{phv}{bx}{n}{0.8cm} % Font for author name: Helvetica 0.8 cm

\def\currentYear{2021}

\usepackage{scrhack}
\usepackage[headsepline]{scrlayer-scrpage} % Provides headers and footers configuration
\pagestyle{scrheadings} % Print the headers and footers on all pages
\clearscrheadfoot % Clean old definitions if they exist

\automark[chapter]{chapter}
\ohead{\headmark} % Prints outer header

\setlength{\headheight}{25pt} % Makes the header take up a bit of extra space for aesthetics
\addtokomafont{headsepline}{\color{lightgray}} % Colors the rule under the header light gray

\ofoot[\normalfont\normalcolor{\thepage\ |\  \pageref{LastPage}}]{\normalfont\normalcolor{\thepage\ |\  \pageref{LastPage}}} % Creates an outer footer of: "current page | total pages"

% These lines make it so each new lab day directly follows the previous one i.e. does not start on a new page - comment them out to separate lab days on new pages
\makeatletter
\patchcmd{\addchap}{\if@openright\cleardoublepage\else\clearpage\fi}{\par}{}{}
\makeatother
\renewcommand*{\chapterpagestyle}{scrheadings}

% These lines make it so every figure and equation in the document is numbered consecutively rather than restarting at 1 for each lab day - comment them out to remove this behavior
\usepackage{chngcntr}
\counterwithout{figure}{labday}
\counterwithout{equation}{labday}

% Hyperlink configuration
\usepackage[
    pdfauthor={Hannah Grady, Nicholas Nauman}, % Your name for the author field in the PDF
    pdftitle={Laboratory Journal}, % PDF title
    pdfsubject={labNotebookSeniorProject3}, % PDF subject
    bookmarksopen=true,
    linktocpage=true,
    urlcolor=linkcolor, % Color of URLs
    citecolor=linkcolor, % Color of citations
    linkcolor=linkcolor, % Color of links to other pages/figures
    backref=page,
    pdfpagelabels=true,
    plainpages=false,
    colorlinks=true, % Turn off all coloring by changing this to false
    bookmarks=true,
    pdfview=FitB]{hyperref}

\usepackage[stretch=10]{microtype} % Slightly tweak font spacing for aesthetics

\usepackage[framed,numbered,autolinebreaks,useliterate]{mcode}
\usepackage{todonotes}

% This package is for plotting graphs
\usepackage{pgfplots}

%\setlength\parindent{0pt} % Uncomment to remove all indentation from paragraphs

%----------------------------------------------------------------------------------------
%    DEFINITION OF EXPERIMENTS
%----------------------------------------------------------------------------------------

% Template: \newexperiment{<abbrev>}[<short form>]{<long form>}
% <abbrev> is the reference to use later in the .tex file in \experiment{}, the <short form> is only used in the table of contents and running title - it is optional, <long form> is what is printed in the lab book itself

\newexperiment{example}[Example experiment]{This is an example experiment}
\newexperiment{example2}[Example experiment 2]{This is another example experiment}
\newexperiment{example3}[Example experiment 3]{This is yet another example experiment}

\newsubexperiment{subexp_example}[Example sub-experiment]{This is an example sub-experiment}
\newsubexperiment{subexp_example2}[Example sub-experiment 2]{This is another example sub-experiment}
\newsubexperiment{subexp_example3}[Example sub-experiment 3]{This is yet another example sub-experiment}

%----------------------------------------------------------------------------------------
\newcommand{\HRule}{\rule{\linewidth}{0.5mm}} % Command to make the lines in the title page

\setlength\parindent{0pt} % Removes all indentation from paragraphs

\begin{document}

%----------------------------------------------------------------------------------------
%    TITLE PAGE
%----------------------------------------------------------------------------------------
%\frontmatter % Use Roman numerals for page numbers

%\begin{center}

%

\title{
\begin{center}
\href{http://www.bradley.edu}{\includegraphics[height=0.5in]{figs/logoBU1-Print}}
\vskip10pt
\HRule \\[0.4cm]
{\Huge \bfseries Laboratory Notebook \\[0.5cm] \Large Hardware-in-Loop Plant Modeling}\\[0.4cm] % Degree
\HRule \\[1.5cm]
\end{center}
}
\author{ Hannah Grady, Nicholas Nauman \\ \\\Large hgrady@mail.bradley.edu, nnauman@mail.bradley.edu} % Your name and email address
\date{Beginning September 14, \currentYear} % Beginning date
\maketitle


%\maketitle % Title page

\printindex
\tableofcontents % Table of contents
\newpage % Start lab look on a new page

\begin{addmargin}[0cm]{0cm} % Makes the text width much shorter for a compact look

\pagestyle{scrheadings} % Begin using headers

%----------------------------------------------------------------------------------------
%    LAB BOOK CONTENTS
%----------------------------------------------------------------------------------------

\labday{Wednesday, September 15, \currentYear}
\experiment{Meeting Minutes}
% Place your initial below
HG

Dr. Miah, Mr. Guetz, Mr. Buckner, and I discussed the project design form and work statement. Nick was not present. Dr. Miah, Nick, and myself will be recieving documents and diagrams from AutonomouStuff around September 22nd which will add further detail to our project. We also went over project logistics and discussed using GitHub for project work unless the information became confidential. Project deliverables will be finalized at the next meeting, and weekly meetings will be conducted via Zoom. Nick and I will have at least six hours of lab work each week, not including extra research. Agendas will be sent out before each meeting.

%-------------------------------------------

\labday{Thursday, September 16, \currentYear}
\experiment{Meeting Minutes}
% Place your initial below
HG\\

Dr. Miah, Nick, and I met to recap some of the aspects of the meeting on 9/15 and discuss what we would be doing as students for this project. It is important to bring a notebook to all meetings in order to record the meeting minutes, and that all agendas and meeting minutes should be recorded in the lab notebook. Meetings will be on Thursdays from 4-5pm and Dr. Miah will send out meeting invitations to everyone. Lab work will take place on Tuesdays and Thursdays from 4-7pm. Dr. Miah will join these lab sessions either in-person or virtually. We will be using camel case convention for the naming of all folders and files, and Dr. Miah will provide the templates we need. Nick and I will send our GitHub information to Dr. Miah so we can access the senior project GitHub, and Dr. Miah will share a Google Drive folder with us. We discussed the format for lab notebook entries, and Nick and I will need to install Inkscape and TeX Live. 

% -------------------------------------------

\labday{Thursday, September 23, \currentYear}
\experiment{Meeting Minutes}
% Place your initial below
HG\\

Dr. Miah, Mr. Guetz, Mr. Buckner, Nick, and I met to further discuss the project. Mr. Guetz provided block diagrams for the systems that we will be modeling. Everything highlighted in red is related to the vehicle, while everything in blue is related to firmware. The torque voltages range from 1.8-3.8V and based on the voltage the steering wheel will turn either left or right. AutonomouStuff already has a steering controller, but it has trouble with small steering angles. Some deliverables we discussed were having a Simulink model that mathematically aligns with data along with a report, test plans for data collection, small steering angle and mathematical model, and results. Once we get raw data from an actual vehicle we will then come up with a Simulink model. A tentative schedule we discussed was spending the first semester gathering data and modeling that data, and then conducting experiments during the second semester. We might go out to AutonomouStuff to collect data, right now we need to start thinking about what type of data we need to collect and formulate a test plan. When we do this depends on vehicle availability, and AutonomouStuff will let us know what days the vehicle will be available. Dr. Miah, Nick, and myself were granted access to the BitBucket repository for this senior project by AutonomouStuff. This is where all documents from this meeting will go. At the end of the meeting Nick asked if there were any ranges we needed to focus on for data collection. The answer is that we need to look at speeds, with 20mph being the max limit. 

% -------------------------------------------
%-------------------------------------------

\experiment{Background study}

HG\\
During the lab, I worked on reading and taking notes of the different sections of the "Get Started with Sytem Identification Toolbox" documentation in MathWorks. I was able to read most of the documentation in the "About System Identification" section, but I will have to finish reading the rest of it and completing the tutorials over the weekend. 

\vspace*{12pt}

NN\\
Today during the lab time, I researched how the Mathworks System Identification Toolbox allows users to create mathematical models of dynamic systems from measured input and output data using MATLAB functions and Simulink in either time-domain or frequency-domain. The research was completed using the following documentation: \href{https://www.mathworks.com/help/ident/getting-started-1.html} \\
There are multiple methods for obtaining identification, but I believe the most effective method for this project will be to capture important system dynamics and organize the data into variables. Once the data is assigned to variables, it can be imported into the System Identification app or set as a data object that can be used to estimate a model from the command line. The System Identification Toolbox software supports time-domain and frequency-domain data for creating linear models and only time-domain data for creating nonlinear models. The data can be single or multiple inputs and outputs that are either real or complex. Once the data is prepared and ran through the System Identification Toolbox, there are functions that are used for either Linear Model Estimation, Nonlinear Model Estimation, or Grey-Box Model Estimation methods. Once the model equations are built, there are options for model validation (comparing experimental data with expected model data), Model Analysis (Model Transformation, Data Extraction, Simulation and Prediction, and Response Computation), and Time Series Analysis.

% -------------------------------------------


\labday{Tuesday, September 28, \currentYear}
\experiment{Background Study}
% Place your initial below
HG\\
During part of the lab time we met with Dr. Miah. We discussed how a folder called "labnotebook" was added to BitBucket. The Virtual Desktop Matlab application doesn't have the System Identification toolbox, Dr. Miah will ask about getting it added to the virtual desktop. We talked about what we have been working on in lab so far. Today I mainly read through some System Identification Toolbox documentation and found articles for the literature review. 

NN\\
During the lab time after meeting with Dr. Miah, I worked on reading documentation on the Matlab System Identification Toolbox and looking at examples of using the System Identification Toolbox. Also, I started creating the block diagrams for the vehicle subsystems that we will be collecting data for and modeling. I got the steering system, acceleration system, shift system, and speed control system block diagrams created. 

% -------------------------------------------

\labday{Thursday, September 30, \currentYear}
\experiment{Meeting Minutes}
% Place your initial below
Agenda:
\begin{itemize}
	\item Discuss a plan for data collection of the Lexus car platform
	\item Discuss the logistics of when data collection may occur
\end{itemize}

NN\\
In today's meeting, Dr. Miah, Mr. Guetz, Hannah, and I discussed the ranges of the inputs and outputs of the steering model so a data collection test plan could be created. Also, we discussed the logistics of collecting this data. The Lexus platform will not be available for testing for at least two weeks. This should allow Mr. Guetz enough time to get the firmware ready for collecting data.

\experiment{Background study}

HG\\
Today during the lab time I started reading through documents that could be used in the literature review section of our proposal and summarizing them. I also watched some videos on the MathWorks website about using the System Identification Toolbox. 


%-------------------------------------------

\labday{Tuesday, October 5, \currentYear}
\experiment{Background Study}
% Place your initial below
HG\\
During the lab time Nick and I started working on the System Level Functional
Requirements presentation. I added to the Problem Statement, Literature Review,
and References slides. We also met with Dr. Miah to go over the expectations for
the presentation and so he could give us feedback on our draft of the System
Level Functional Requirements document. Dr. Miah also pushed the changes we made
to some of the documents to the GitHub repository we have for this project.

NN\\
During the lab time, I added the Introduction, System Inputs/Outputs, System Requirements, and Concluding Remarks sections of the System Level Functional Requirements presentation. 

%--------------------------------------------------------------------------------------------

\labday{Thursday, October 7, \currentYear}
\experiment{Data Collection}
% Place your initial below
HG\\
Today Dr. Miah, Nick, and I went to AutonomouStuff in Morton to collect data. We met Mr. Guetz and Mr. Buckner there, and they helped us sign in and showed us where the vehicles were kept. Dr. Miah, Nick, Mr. Guetz, and I went out to the Lexus and used that for testing. Dr. Miah was the safety driver while Mr. Guetz used a computer to give commands to the vehicle. Nick and I took notes. There is a red button to shut down the system, and the safety driver can also turn the steering wheel or step on the gas in case anything goes wrong. We started measuring data for the steering model first. At rest the steering angle is zero, and if the driver tries to manually move the steering wheel while the software is engaged, the torque voltages will fight them. Turning the wheel counterclockwise corresponds to positive radians, while clockwise corresponds to negative radians. The wheel angle is proportional to the steering angle. Dr. Miah turned the steering wheel all the way to lock and back twice to collect that data. We then moved onto collecting data at 0.1 radians to the left and then to the right multiple times. From there we moved onto pi/2 radian turns at 5mph. We also collected data for multiple steering angles at 10mph. We then kept collecting data, keeping the car at a speed between 10-11mph. Mr. Guetz explained that the car can also be controlled using a joystick, but that tends to be jittery. Data can also be collected while the car is being driven manually, and we ended up collecting data while the car was being controlled manually and automatically. From the computer, Mr. Guetz requests a position, and the car produces a single torque voltage that gets split into two torque voltages. Torque is the amount of assist applied to the rack. Once we had collected the steering data, we then moved onto collecting brake data. We collected that data while parked, and gathered data while the car was being operated manually and automatically. In automatic mode, there is no visible pedal change, since everything happens internally. We did two rounds of testing, during the first round we collected data manually at a range of 0-93 and then automatically from 0-100. For the second round we tested the brakes manually from 0-98 and automatically from 0-100. We got the data from this in terms of voltages. At the range of 0-75, the brake pressure is low, and you reach a maximum pressure as you approach 100. Afterwards we collected data for our acceleration model. It was similar to the data collection for the brake model in that the vehicle was put in park, only this time the acceleration pedal was pushed. We did two rounds of data collection in manual and automatic. The first round tested the acceleration from 0-100 for the vehicle in manual and automatic, and did the same thing for the second round. The input to the system is the voltages and the output is the pedal position. We will come back to AutonomouStuff later to collect data for the shift and gear models. 

NN\\
After completing the testing as described above, we returned to the garage to ask any remaining questions. I asked Mr. Guetz about the hardware setup for each system we are developing models for. He explained that the hardware implementation each similar for each system. The system is connected to the PACMod which is then connected to a CAN case, which connects to the laptop. The laptop can be used to log this CAN data, which is how we collected the data we will be analyzing. 

%--------------------------------------------------------------------------------------------
\labday{Thursday, October 14, \currentYear}
\experiment{Background Study}
% Place your initial below
HG\\
At the beginning of the lab time, Nick and I met with Dr. Miah. He let us know that he had uploaded the pictures from our trip to AutonomouStuff. Nick mentioned that he had started using the System Identification Toolbox with the data we collected last Thursday. We went over a block diagram for the steering model. There are two torque voltages A, and B, so the model will have to be updated. The controller deals with the error between the desired steering angle (which Eric was setting during data collection) and the actual steering angle. We discussed how we need to understand the physical significance of A and B and what they do. We went over the data that Nick already had open in Matlab. The commanded value might not be a percentage, and instead may be in radians. We will have to ask Eric about this. The steering RPT Output value is the actual steering angle. The actual steering wheel angle is based on output voltages A and B, and using these values will help us to identify our two-input two-output system. Nick and I will have to do some research on what order system reliably gives a good model. We will also have to do research and see how people come up with different models based on a two-input, one output system. We will include any research in our proposal. Dr. Miah will ask Mr. Mattus about having the System Identification Toolbox on the lab computers. 

NN\\
During the lab time, Dr. Miah, Hannah, and I discussed the data we collected during our trip to AutonomouStuff. Dr. Miah showed us how to develop a block diagram to describe the behavior of the steering model. This allowed us to easily select which data we needed to use as the inputs and outputs for using in Matlab. With these details, I started working on trying to develop a model of the steering motor using the data collected from AutonomouStuff. I ran into some problems with not add my data correctly so I researched some articles that explained how to process the data.

%-------------------------------------------

\labday{Tuesday, October 19, \currentYear}
\experiment{Background Study}
% Place your initial below
HG\\
At the beginning of lab Nick and I met with Dr. Miah to discuss what we have been working on. We talked about the steering model and how the system had two inputs (torque voltages A and B) and one output (Actual SW angle). We discussed the possible options for identifying a model for the steering system using the System Identification Toolbox. One option is to use A and B as  the inputs and the commanded (desired) steering angle as the output. We talked about creating an agenda for Thursday's meeting that includes questions we have for Erik. This agenda will be placed in BitBucket along with our modified diagrams. The System Level Functional Requirements Document must be finalized by October 21st, along with the presentation. Nick and I will work on this outside of lab time. During the lab time Dr. Miah wanted us to work with the data and find examples showing system modeling of MISO systems using the toolbox. Nick has already started working with the Data before our meeting today. He has generated a transfer function, and wants to tune the generated parameters. We will try all possible options and see what gives the best fit and validation data for a model. Since we will have two inputs, when both inputs are activated, the ultimate output is the superimposed result. The inputs should be coupled, so we will need to modify the way we are simulating. 

During the lab time I worked on finding examples of system identification of MISO systems. I was able to find a couple related documents, including one Matlab tutorial. I worked through this tutorial during the lab time. 
\begin{itemize}
	\item \href{https://www.mathworks.com/help/ident/gs/identify-linear-models-using-the-command-line.html}{Example}
\end{itemize}

NN\\
During the lab time, Dr. Miah, Hannah, and I discussed the data that we collected from AutonomouStuff. We discussed the need to get clarification from Erik and Joe about what units were being used for the actual steering angle, commanded steering angle, and reference steering angle. We also discussed getting clarification on what the non-linearities actual mean to the system. After the discussion, I worked on manipulation the data in the System Identification Toolbox to try and get a model that closely follows the system output. 

%---------------------------------------------------------

\labday{Thursday, October 21, \currentYear}
\experiment{Meeting Minutes}
% Place your initial below
HG\\
I arrived late to today's meeting with the AutonomouStuff since my class doesn't get out until 4:15pm. When I entered the meeting Nick was sharing his screen and the data that he had so far. He mentioned that the parameters needed to be tuned up so that the desired steering wheel angle matched the actual steering wheel angle. The current model he had wasn't very good at predicting the steering angle but it got better as time went on. We will have to try tuning a model to handle small steering wheel angles and then apply that model to the larger steering wheel angle data to see if it works. Dr. Miah's plan is to create data sets out of the data that we have and then generate random test sets. We will then isolate some data sets where the response is small steering wheel angles and make test data. Dr. Miah drew a new steering wheel model block diagram and then two options. The first option was to use the output torque voltages A and B from the steering wheel model as inputs to a predictive steering model, with the output being the actual steering wheel angle. The second option was to take the torque voltages A and B from the controller output as our predictive model's inputs and then have the output be the commanded steering wheel angle. The problem is mainly coming up with a model and then tuning the parameters in such a way so the real data matches with the predictive data. After getting a model, it will be tested with AutonomouStuff's controller to see if they can get better control. Our model will have to have the non-linearities seen in the real world. Mr. Guetz mentioned that a small steering wheel angle is considered to be between -10 and 10 degrees. We also talked about how a small steering angle means that from time to time the steering angle varies very little. Mr. Guetz explained that the non-linearities are larger in smaller angles, especially when the car is trying to stay in one lane. These small non-linearities cause the car to swerve back and forth. He then drew us a diagram to help illustrate the non-linearities, which occur when going to the center. We will need to add a section on non-linearities to our report. We will need to predict our model in such a way that it will respond to small non-linear angles faster than the existing system. As we go along we will explore a deep reinforcement learning algorithm. Erik will run through the data and make sure that it is okay. He will then send it to us by Monday. For now we will work with the clean data and then work with the rest of the data.

\experiment{Background Study}
% Place your initial below
HG\\ 
After our meeting with AutonomouStuff, Nick, Dr. Miah, and I met. Dr. Miah told us to write everything and place all diagrams and plots in the lab notebook and project proposal. Going forward we will be working on the proposal template. We will start using GitHub for all document templates so Nick and I can edit simultaneously. Nick and I divided up the sections of the proposal to work on and I looked for examples of using system identification for MISO or MIMO systems. 

NN\\
Dr. Miah, Hannah, and I met with Joe and Erik to discuss clarifications with the units of the steering angle since there was some confusion from an email Erik sent. We also got some clarification on what the system non-linearities were and how the affected the steering. Erik also provided some context to how the models would be used once we completed them. While we wait for Erik to send an email with what is the bad data in the data sets, Dr. Miah advised us to do more research on using the System Identification Toolbox for modeling MISO systems and examples we could run through. These are some of the articles I found:
\begin{itemize}
	\item \href{https://www.mathworks.com/help/ident/ug/dealing-with-multi-variable-systems-identification-and-analysis.html}{Example 1}
	\item \href{https://www.mathworks.com/help/ident/ug/modal-analysis-of-a-flexible-flying-wing-aircraft.html}{Example 2}
	\item \href{https://www.mathworks.com/help/ident/gs/system-identification-workflow.html}{Workflow}
	\item \href{https://www.mathworks.com/help/ident/model-validation-basics.html}{Validation Basics}
\end{itemize}


%-------------------------------------------

\labday{Tuesday, October 26, \currentYear}
\experiment{System Identification}
% Place your initial below
HG\\
Dr. Miah, Nick, and I met at the beginning of the lab time to discuss our progress. I had read a paper over the weekend that mentioned modeling a MISO system, and so during the lab time I finished summarizing it and placed that summary in the proposal document. There is an error that keeps popping up that I tried to resolve, but after doing some research, I still wasn't able to fix the problem. I will have to email Dr. Miah about this. It did not affect my ability to add stuff to the proposal. I also wrote a mathematical equation for the summary and also placed that in the project proposal. I then found another example in Matlab about dealing with MIMO systems and estimating models, so I worked on that during the lab time. 
\begin{itemize}
	\item \href{https://www.mathworks.com/help/ident/ug/dealing-with-multi-variable-systems-identification-and-analysis.html}{Example}
\end{itemize}

NN\\
During the lab time today, Dr. Miah advised Hannah and I to work through some examples for using the Matlab System Identification Toolbox, specifically for MISO or MIMO systems so that we can verify we are working through the process correctly. I worked through the following examples and when I was confused by what a certain function did, I looked at the documentation to understand it better. 
\begin{itemize}
	\item \href{https://www.mathworks.com/help/ident/ug/dealing-with-multi-variable-systems-identification-and-analysis.html}{Example 1}
	\item \href{http://www.ece.northwestern.edu/local-apps/matlabhelp/toolbox/ident/ch3tut44.html}{Example 2}
\end{itemize}

Also, Dr. Miah announced that the Matlab System Identification Toolbox was installed on computers 9 and 10 in the Senior Lab.

\labday{Thursday, October 28, \currentYear}
\experiment{Meeting Minutes}
% Place your initial below
NN\\
During the lab time, Dr. Miah, Erik, Hannah, and I met to discuss getting clean data to work with for the steering data we collected. Erik said he could get us clean data by Monday, and I would send a reminder email if he forgot to upload the data. After the meeting, Hannah and I worked on summarizing the Matlab System Identification Toolbox examples we found into the proposal document. After that, we worked on trying to use the System Identification Toolbox on the steering data we collected.

\experiment{Background Study}
% Place your initial below
HG\\
After our meeting with AutonomouStuff, Nick, Dr. Miah, and I met to discuss what we would be working on for the rest of the lab time. Dr. Miah reminded us to not export any whitespace from Inkscape. We also talked about adding the matlab examples we worked on to the lab notebook. From now on, we will be pulling the repository from GitHub before starting work on any documents. During the lab time I worked with the data that we had in Matlab and also started summarizing the matlab examples to be placed in the project proposal. 

NN\\


\labday{Tuesday, November 2, \currentYear}
\experiment{Meeting Minutes}
% Place your initial below
HG\\


NN\\
During the lab time, Dr. Miah, Hannah, and I met to discuss what items should be worked on. There is a draft proposal document due by the end of today that Dr. Miah reviewed and advised us to submit. For the rest of the lab time, Hannah and I worked on developing models of the steering system in the Matlab System Identification Toolbox so the models could be reviewed with AutonomouStuff on Thursday.

%-------------------------

\labday{Thursday, November 4, \currentYear}
\experiment{Meeting Minutes}
% Place your initial below
HG\\
At the beginning of lab Dr. Miah, Erik, Nick, and I met to talk about our progress. Going from the meeting I worked on modeling the manual and by-wire steering angle systems. 

NN\\
During the lab time, Dr. Miah, Erik, Hannah, and I met to discuss the models we developed to get feedback. Hannah presented her state-space model that had a best fit percentage of 59. I presented my transfer function model that had a best fit percentage of 96. Erik stated that these models still did not track close enough to the non-linearities, but were accurate almost every else. We talked about each of the logs. Logs 1 and 3 were manual steering angle data, while Log 2 was a commanded steering angle (By-Wire method). Erik recommended running two different training sessions: one for the By-Wire method and one for the manual method. He recommended this because for the By-Wire method, the electronic control unit (ECU) gets rid of the input torque voltages by open circuiting the motors and instead sending output torque voltages from the controller to the motors. For the rest of the lab time, I worked on running the separate training sessions for the steering system model and then started preparing data for the acceleration model.

%-------------------------

\labday{Tuesday, November 9, \currentYear}
\experiment{Lab Time}
% Place your initial below
HG\\
At the beginning of the lab time Dr. Miah and I met. We briefly went over the proposal presentation and he mentioned that we have to have all of the figures in the presentation in the proposal document. He will work on getting the Gantt chart template added to the presentation so we can use it in the Timeline and Milestones section. During the lab today he asked me to work on getting the raw data and plotting the actual and desired steering wheel angles along with the measured and controller A and B voltage values. Afterwards I need to place these plots into the proposal document. He also corrected some of the errors that were showing up in the proposal document. During lab I worked on generating these plots, and I also started working on modeling the acceleration subsystem. I have some questions about modeling it so I didn't get as far as I had hoped. 

NN\\
I was unable to attend the meeting during lab time due to a conflict. I worked on modeling the acceleration subsystem on Wednesday, November 10. I was able to derive models that are very accurate for both by-wire and manual methods. I have attached the results in the proposal presentation.

%-------------------------

\labday{Thursday, November 11, \currentYear}
\experiment{Meeting Minutes}
% Place your initial below
HG\\
I arrived late to the meeting due to a doctor's appointment. Dr. Miah, Erik, Joe and I met at the beginning of the lab time to go over the presentation and the steering system model. During the lab time I worked on developing a model for the brake subsystem.  

NN\\
Dr. Miah, Erik, Joe, Hannah, and I met to discuss the models we had developed so far. I presented the models developed for the steering system and acceleration system for discussion and feedback. Dr. Miah and Joe recommended adding more labels to the graph for giving a clear message of what the plot means. We also looked at the error plots to determine if the error was within an acceptable range and to determine the behavior (quick spikes or smooth plot). Also, we discussed implementing our developed models on the AutonomouStuff HIL bench to determine the accuracy of the model with new data. Erik suggested adding the developed model into the HIL bench and running it in parallel with the current model AutonomouStuff uses and we can compare the effectiveness during the model. For the lab time, I worked on adding labels and other details to the figures of the developed models of the steering and acceleration system. I also added the mathematical equations for these models.

%-------------------------

\labday{Tuesday, November 16, \currentYear}
\experiment{Lab Time}
% Place your initial below
HG\\
At the beginning of lab Dr. Miah, Nick, and I went over the proposal presentation. He offered suggestions on what to change, such as making the graphs larger and adding a timeline to the timeline and milestones slide. He also mentioned that all changes we make and any pictures we add should also be added to the proposal document. During the lab time I worked on the presentation by adding in a MATLAB example for the Preliminary Work slide and adding a break to the references slide. I also made some small changes to the title and concluding remarks slides. I then added a timeline for the timeline and milestones slide. Nick and I also divided up what slides we would present on Thursday. 

NN\\
After reviewing the suggestions Dr. Miah provided, I worked on adding figures for the simulation results, adding tables for the coefficients of the model transfer functions, adding an experimental setup picture to that slide, and making some edits to wording on some of the slides I would be presenting.

%-------------------------

\labday{Thursday, November 18, \currentYear}
\experiment{Meeting Minutes}
% Place your initial below
HG\\
At the beginning of lab, Nick and I gave our proposal presentation to Dr. Miah, Erik, and Joe. Once we were done presenting they gave us feedback on how we could improve our presentation. During the lab time I worked on editing the parts list, concluding remarks, and problem statement slides. 

NN\\
After Hannah and I gave our proposal presentation to Dr. Miah, Erik, and Joe and they provided feedback, the rest of the lab time was spent updating the slide deck based on the feedback. I fixed the figures of the simulation results and the wording of the experimental setup. After that, I worked on adding some of these details to the proposal document.

%-------------------------

\labday{Tuesday, November 23, \currentYear}
\experiment{Meeting Minutes}
% Place your initial below
HG\\
During the lab time I worked on making the font for the inputs and outputs of the steering, brake, and speed model block diagrams. The acceleration, shift, and speed control model block diagrams didn't show up when I opened them in Inkscape. I then finished reading through the proposal document and made some small changes to it. 

%--------------------------------------------------------------------------------------------

\end{addmargin}

%----------------------------------------------------------------------------------------
%    BIBLIOGRAPHY
%----------------------------------------------------------------------------------------


\bibliographystyle{plain}
\bibliography{bib/references.bib}


% \begin{thebibliography}{9}



% \end{thebibliography}

%----------------------------------------------------------------------------------------

\end{document}


%%% Local Variables:
%%% mode: latex
%%% TeX-master: t
%%% End:
