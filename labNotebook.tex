%%%%%%%%%%%%%%%%%%%%%%%%%%%%%%%%%%%%%%%%
% Compact Laboratory Book
% LaTeX Template
% Version 1.0 (4/6/12)
%
% This template has been downloaded from:
% http://www.LaTeXTemplates.com
%
% Original author:
% Joan Queralt Gil (http://phobos.xtec.cat/jqueralt) using the labbook class by
% Frank Kuster (http://www.ctan.org/tex-archive/macros/latex/contrib/labbook/)
%
% License:
% CC BY-NC-SA 3.0 (http://creativecommons.org/licenses/by-nc-sa/3.0/)
%
% Important note:
% This template requires the labbook.cls file to be in the same directory as the
% .tex file. The labbook.cls file provides the necessary structure to create the
% lab book.
%
% The \lipsum[#] commands throughout this template generate dummy text
% to fill the template out. These commands should all be removed when 
% writing lab book content.
%
% HOW TO USE THIS TEMPLATE 
% Each day in the lab consists of three main things:
%
% 1. LABDAY: The first thing to put is the \labday{} command with a date in 
% curly brackets, this will make a new section showing that you are working
% on a new day.
%
% 2. EXPERIMENT/SUBEXPERIMENT: Next you need to specify what 
% experiment(s) and subexperiment(s) you are working on with a 
% \experiment{} and \subexperiment{} commands with the experiment 
% shorthand in the curly brackets. The experiment shorthand is defined in the 
% 'DEFINITION OF EXPERIMENTS' section below, this means you can 
% say \experiment{pcr} and the actual text written to the PDF will be what 
% you set the 'pcr' experiment to be. If the experiment is a one off, you can 
% just write it in the bracket without creating a shorthand. Note: if you don't 
% want to have an experiment, just leave this out and it won't be printed.
%
% 3. CONTENT: Following the experiment is the content, i.e. what progress 
% you made on the experiment that day.
%
%%%%%%%%%%%%%%%%%%%%%%%%%%%%%%%%%%%%%%%%%

%----------------------------------------------------------------------------------------
%    PACKAGES AND OTHER DOCUMENT CONFIGURATIONS
%----------------------------------------------------------------------------------------                               

% \UseRawInputEncoding

\documentclass[fontsize=11pt, % Document font size
                             paper=letter, % Document paper type
                             %twoside, % Shifts odd pages to the left for easier reading when printed, can be changed to oneside
                             openany, % chapters can start on any page
                             captions=tableheading,
                             index=totoc,
                             hyperref]{labbook}

%\documentclass[idxtotoc,hyperref,openany]{labbook} % 'openany' here removes the
  
\usepackage[bottom=10em]{geometry} % Reduces the whitespace at the bottom of the page so more text can fit

\usepackage[english]{babel} % English language
\usepackage{lipsum} % Used for inserting dummy 'Lorem ipsum' text into the template

\usepackage[utf8]{inputenc} % Uses the utf8 input encoding
\usepackage[T1]{fontenc} % Use 8-bit encoding that has 256 glyphs

\usepackage[osf]{mathpazo} % Palatino as the main font
\linespread{1.05}\selectfont % Palatino needs some extra spacing, here 5% extra
\usepackage[scaled=.88]{beramono} % Bera-Monospace
\usepackage[scaled=.86]{berasans} % Bera Sans-Serif

\usepackage{booktabs,array} % Packages for tables

\usepackage{amsmath} % For typesetting math
\usepackage{graphicx} % Required for including images
\usepackage{etoolbox}
\usepackage[norule]{footmisc} % Removes the horizontal rule from footnotes
\usepackage{lastpage} % Counts the number of pages of the document
\usepackage{float}

\usepackage[ruled, vlined, linesnumbered]{algorithm2e} % For algorithms


\usepackage[dvipsnames]{xcolor}  % Allows the definition of hex colors
\usepackage{epstopdf}
\epstopdfsetup{suffix={}}
\definecolor{titleblue}{rgb}{0.16,0.24,0.64} % Custom color for the title on the title page
\definecolor{linkcolor}{rgb}{0,0,0.42} % Custom color for links - dark blue at the moment

\addtokomafont{title}{\Huge\color{titleblue}} % Titles in custom blue color
\addtokomafont{chapter}{\color{OliveGreen}} % Lab dates in olive green
\addtokomafont{section}{\color{Sepia}} % Sections in sepia
\addtokomafont{pagehead}{\normalfont\sffamily\color{gray}} % Header text in gray and sans serif
\addtokomafont{caption}{\footnotesize\itshape} % Small italic font size for captions
\addtokomafont{captionlabel}{\upshape\bfseries} % Bold for caption labels
\addtokomafont{descriptionlabel}{\rmfamily}
\setcapwidth[c]{10cm} % Center align caption text
\setkomafont{footnote}{\sffamily} % Footnotes in sans serif

\deffootnote[4cm]{4cm}{1em}{\textsuperscript{\thefootnotemark}} % Indent footnotes to line up with text

\DeclareFixedFont{\textcap}{T1}{phv}{bx}{n}{1.5cm} % Font for main title: Helvetica 1.5 cm
\DeclareFixedFont{\textaut}{T1}{phv}{bx}{n}{0.8cm} % Font for author name: Helvetica 0.8 cm

\def\currentYear{2021}

\usepackage{scrhack}
\usepackage[headsepline]{scrlayer-scrpage} % Provides headers and footers configuration
\pagestyle{scrheadings} % Print the headers and footers on all pages
\clearscrheadfoot % Clean old definitions if they exist

\automark[chapter]{chapter}
\ohead{\headmark} % Prints outer header

\setlength{\headheight}{25pt} % Makes the header take up a bit of extra space for aesthetics
\addtokomafont{headsepline}{\color{lightgray}} % Colors the rule under the header light gray

\ofoot[\normalfont\normalcolor{\thepage\ |\  \pageref{LastPage}}]{\normalfont\normalcolor{\thepage\ |\  \pageref{LastPage}}} % Creates an outer footer of: "current page | total pages"

% These lines make it so each new lab day directly follows the previous one i.e. does not start on a new page - comment them out to separate lab days on new pages
\makeatletter
\patchcmd{\addchap}{\if@openright\cleardoublepage\else\clearpage\fi}{\par}{}{}
\makeatother
\renewcommand*{\chapterpagestyle}{scrheadings}

% These lines make it so every figure and equation in the document is numbered consecutively rather than restarting at 1 for each lab day - comment them out to remove this behavior
\usepackage{chngcntr}
\counterwithout{figure}{labday}
\counterwithout{equation}{labday}

% Hyperlink configuration
\usepackage[
    pdfauthor={Hannah Grady, Nicholas Nauman}, % Your name for the author field in the PDF
    pdftitle={Laboratory Journal}, % PDF title
    pdfsubject={labNotebookSeniorProject3}, % PDF subject
    bookmarksopen=true,
    linktocpage=true,
    urlcolor=linkcolor, % Color of URLs
    citecolor=linkcolor, % Color of citations
    linkcolor=linkcolor, % Color of links to other pages/figures
    backref=page,
    pdfpagelabels=true,
    plainpages=false,
    colorlinks=true, % Turn off all coloring by changing this to false
    bookmarks=true,
    pdfview=FitB]{hyperref}

\usepackage[stretch=10]{microtype} % Slightly tweak font spacing for aesthetics

\usepackage[framed,numbered,autolinebreaks,useliterate]{mcode}
\usepackage{todonotes}

% This package is for plotting graphs
\usepackage{pgfplots}

%\setlength\parindent{0pt} % Uncomment to remove all indentation from paragraphs

%----------------------------------------------------------------------------------------
%    DEFINITION OF EXPERIMENTS
%----------------------------------------------------------------------------------------

% Template: \newexperiment{<abbrev>}[<short form>]{<long form>}
% <abbrev> is the reference to use later in the .tex file in \experiment{}, the <short form> is only used in the table of contents and running title - it is optional, <long form> is what is printed in the lab book itself

\newexperiment{example}[Example experiment]{This is an example experiment}
\newexperiment{example2}[Example experiment 2]{This is another example experiment}
\newexperiment{example3}[Example experiment 3]{This is yet another example experiment}

\newsubexperiment{subexp_example}[Example sub-experiment]{This is an example sub-experiment}
\newsubexperiment{subexp_example2}[Example sub-experiment 2]{This is another example sub-experiment}
\newsubexperiment{subexp_example3}[Example sub-experiment 3]{This is yet another example sub-experiment}

%----------------------------------------------------------------------------------------
\newcommand{\HRule}{\rule{\linewidth}{0.5mm}} % Command to make the lines in the title page

\setlength\parindent{0pt} % Removes all indentation from paragraphs

\begin{document}

%----------------------------------------------------------------------------------------
%    TITLE PAGE
%----------------------------------------------------------------------------------------
%\frontmatter % Use Roman numerals for page numbers

%\begin{center}

%

\title{
\begin{center}
\href{http://www.bradley.edu}{\includegraphics[height=0.5in]{figs/logoBU1-Print}}
\vskip10pt
\HRule \\[0.4cm]
{\Huge \bfseries Laboratory Notebook \\[0.5cm] \Large Hardware-in-Loop Plant Modeling}\\[0.4cm] % Degree
\HRule \\[1.5cm]
\end{center}
}
\author{ Hannah Grady, Nicholas Nauman \\ \\\Large hgrady@mail.bradley.edu, nnauman@mail.bradley.edu} % Your name and email address
\date{Beginning September 14, \currentYear} % Beginning date
\maketitle


%\maketitle % Title page

\printindex
\tableofcontents % Table of contents
\newpage % Start lab look on a new page

\begin{addmargin}[0cm]{0cm} % Makes the text width much shorter for a compact look

\pagestyle{scrheadings} % Begin using headers

%----------------------------------------------------------------------------------------
%    LAB BOOK CONTENTS
%----------------------------------------------------------------------------------------

\labday{Wednesday, September 15, \currentYear}
\experiment{Meeting Minutes}
% Place your initial below
HG\\

Dr. Miah, Mr. Guetz, Mr. Buckner, and I discussed the project design form and work statement. Nick was not present. Dr. Miah, Nick, and myself will be recieving documents and diagrams from AutonomouStuff around September 22nd which will add further detail to our project. We also went over project logistics and discussed using GitHub for project work unless the information became confidential. Project deliverables will be finalized at the next meeting, and weekly meetings will be conducted via Zoom. Nick and I will have at least six hours of lab work each week, not including extra research. Agendas will be sent out before each meeting.

%-------------------------------------------

\labday{Thursday, September 16, \currentYear}
\experiment{Meeting Minutes}
% Place your initial below
HG\\

Dr. Miah, Nick, and I met to recap some of the aspects of the meeting on 9/15 and discuss what we would be doing as students for this project. It is important to bring a notebook to all meetings in order to record the meeting minutes, and that all agendas and meeting minutes should be recorded in the lab notebook. Meetings will be on Thursdays from 4-5pm and Dr. Miah will send out meeting invitations to everyone. Lab work will take place on Tuesdays and Thursdays from 4-7pm. Dr. Miah will join these lab sessions either in-person or virtually. We will be using camel case convention for the naming of all folders and files, and Dr. Miah will provide the templates we need. Nick and I will send our GitHub information to Dr. Miah so we can access the senior project GitHub, and Dr. Miah will share a Google Drive folder with us. We discussed the format for lab notebook entries, and Nick and I will need to install Inkscape and TeX Live. 

% -------------------------------------------

\labday{Thursday, September 23, \currentYear}
\experiment{Meeting Minutes}
% Place your initial below
HG\\

Dr. Miah, Mr. Guetz, Mr. Buckner, Nick, and I met to further discuss the project. Mr. Guetz provided block diagrams for the systems that we will be modeling. Everything highlighted in red is related to the vehicle, while everything in blue is related to firmware. The torque voltages range from 1.8-3.8V and based on the voltage the steering wheel will turn either left or right. AutonomouStuff already has a steering controller, but it has trouble with small steering angles. Some deliverables we discussed were having a Simulink model that mathematically aligns with data along with a report, test plans for data collection, small steering angle and mathematical model, and results. Once we get raw data from an actual vehicle we will then come up with a Simulink model. A tentative schedule we discussed was spending the first semester gathering data and modeling that data, and then conducting experiments during the second semester. We might go out to AutonomouStuff to collect data, right now we need to start thinking about what type of data we need to collect and formulate a test plan. When we do this depends on vehicle availability, and AutonomouStuff will let us know what days the vehicle will be available. Dr. Miah, Nick, and myself were granted access to the BitBucket repository for this senior project by AutonomouStuff. This is where all documents from this meeting will go. At the end of the meeting Nick asked if there were any ranges we needed to focus on for data collection. The answer is that we need to look at speeds, with 20mph being the max limit. 

% -------------------------------------------
%-------------------------------------------

\experiment{Background study}

HG\\
During the lab, I worked on reading and taking notes of the different sections of the "Get Started with Sytem Identification Toolbox" documentation in MathWorks. I was able to read most of the documentation in the "About System Identification" section, but I will have to finish reading the rest of it and completing the tutorials over the weekend. 

\vspace*{12pt}

NN\\
Today during the lab time, I researched how the Mathworks System Identification Toolbox allows users to create mathematical models of dynamic systems from measured input and output data using MATLAB functions and Simulink in either time-domain or frequency-domain. The research was completed using the following documentation: \href{https://www.mathworks.com/help/ident/getting-started-1.html} \\
There are multiple methods for obtaining identification, but I believe the most effective method for this project will be to capture important system dynamics and organize the data into variables. Once the data is assigned to variables, it can be imported into the System Identification app or set as a data object that can be used to estimate a model from the command line. The System Identification Toolbox software supports time-domain and frequency-domain data for creating linear models and only time-domain data for creating nonlinear models. The data can be single or multiple inputs and outputs that are either real or complex. Once the data is prepared and ran through the System Identification Toolbox, there are functions that are used for either Linear Model Estimation, Nonlinear Model Estimation, or Grey-Box Model Estimation methods. Once the model equations are built, there are options for model validation (comparing experimental data with expected model data), Model Analysis (Model Transformation, Data Extraction, Simulation and Prediction, and Response Computation), and Time Series Analysis.

% -------------------------------------------


\labday{Tuesday, September 28, \currentYear}
\experiment{Background Study}
% Place your initial below
HG\\
During part of the lab time we met with Dr. Miah. We discussed how a folder called "labnotebook" was added to BitBucket. The Virtual Desktop Matlab application doesn't have the System Identification toolbox, Dr. Miah will ask about getting it added to the virtual desktop. We talked about what we have been working on in lab so far. Today I mainly read through some System Identification Toolbox documentation and found articles for the literature review. 

% -------------------------------------------

\labday{Thursday, September 30, \currentYear}
\experiment{Meeting Minutes}
% Place your initial below
Agenda:
\begin{itemize}
	\item Discuss a plan for data collection of the Lexus car platform
	\item Discuss the logistics of when data collection may occur
\end{itemize}

NN\\
In today's meeting, Dr. Miah, Mr. Guetz, Hannah, and I discussed the ranges of the inputs and outputs of the steering model so a data collection test plan could be created. Also, we discussed the logistics of collecting this data. The Lexus platform will not be available for testing for at least two weeks. This should allow Mr. Guetz enough time to get the firmware ready for collecting data.

\experiment{Background study}

HG\\
Today during the lab time I started reading through documents that could be used in the literature review section of our proposal and summarizing them. I also watched some videos on the MathWorks website about using the System Identification Toolbox. 


%-------------------------------------------

\labday{Tuesday, October 5, \currentYear}
\experiment{Background Study}
% Place your initial below
HG\\
During the lab time Nick and I started working on the System Level Functional
Requirements presentation. I added to the Problem Statement, Literature Review,
and References slides. We also met with Dr. Miah to go over the expectations for
the presentation and so he could give us feedback on our draft of the System
Level Functional Requirements document. Dr. Miah also pushed the changes we made
to some of the documents to the GitHub repository we have for this project.

%--------------------------------------------------------------------------------------------

\end{addmargin}

%----------------------------------------------------------------------------------------
%    BIBLIOGRAPHY
%----------------------------------------------------------------------------------------


\bibliographystyle{plain}
\bibliography{bib/references.bib}


% \begin{thebibliography}{9}



% \end{thebibliography}

%----------------------------------------------------------------------------------------

\end{document}


%%% Local Variables:
%%% mode: latex
%%% TeX-master: t
%%% End:
