% Copyright 2004 by Till Tantau <tantau@users.sourceforge.net>.
%
% In principle, this file can be redistributed and/or modified under
% the terms of the GNU Public License, version 2.
%
% However, this file is supposed to be a template to be modified
% for your own needs. For this reason, if you use this file as a
% template and not specifically distribute it as part of a another
% package/program, I grant the extra permission to freely copy and
% modify this file as you see fit and even to delete this copyright
% notice. 

% \UseRawInputEncoding
\documentclass{beamer}

% There are many different themes available for Beamer. A comprehensive
% list with examples is given here:
% http://deic.uab.es/~iblanes/beamer_gallery/index_by_theme.html
% You can uncomment the themes below if you would like to use a different
% one:
%\usetheme{AnnArbor}
%\usetheme{Antibes}
%\usetheme{Bergen}
%\usetheme{Berkeley}
%\usetheme{Berlin}
%\usetheme{Boadilla}
%\usetheme{boxes}
%\usetheme{CambridgeUS}
%\usetheme{Copenhagen}
%\usetheme{Darmstadt}
%\usetheme{default}
%\usetheme{Frankfurt}
%\usetheme{Goettingen}
%\usetheme{Hannover}
%\usetheme{Ilmenau}
%\usetheme{JuanLesPins}
%\usetheme{Luebeck}
\usetheme{Madrid}
%\usetheme{Malmoe}
%\usetheme{Marburg}
%\usetheme{Montpellier}
%\usetheme{PaloAlto}
%\usetheme{Pittsburgh}
%\usetheme{Rochester}
%\usetheme{Singapore}
%\usetheme{Szeged}
%\usetheme{Warsaw}

\usepackage{pgfgantt}
\usepackage{todonotes}
\usepackage{media9}
\usepackage{subfigure}
\usepackage{booktabs,array}
\usepackage[font=tiny,labelfont=bf]{caption}
\usepackage{tabulary}
\usepackage{caption}
\usepackage{graphicx}
\usepackage{siunitx}
\usepackage{arydshln}
\usepackage{rotating}


% Customize Warsaw color 
\setbeamercolor*{palette primary}{use=structure,fg=white,bg=red!50!black}
\setbeamercolor*{palette secondary}{use=structure,fg=white,bg=red!60!black}
\setbeamercolor*{palette tertiary}{use=structure,fg=white,bg=red!70!black}

% Customize Warsaw block title and background colors
\setbeamercolor{block title}{bg=red!50!black,fg=white}

\setbeamertemplate{navigation symbols}{} % Remove navigation symbols
\setbeamertemplate{bibliography item}{\insertbiblabel}  % insert bibliography numbers instead of symbol
\setbeamertemplate{caption}[numbered] % adds the figure or table number to the caption.

% Set up footer manually
\setbeamertemplate{footline}
{
  \leavevmode%
  \hbox{%
  \begin{beamercolorbox}[wd=.45\paperwidth,ht=2.25ex,dp=1ex,center]{author in head/foot}%
    \usebeamerfont{author in head/foot}\insertshortauthor\hspace*{1em}(\insertshortinstitute)
  \end{beamercolorbox}%
  \begin{beamercolorbox}[wd=.4\paperwidth,ht=2.25ex,dp=1ex,center]{title in head/foot}%
    \usebeamerfont{title in head/foot}\insertshorttitle
  \end{beamercolorbox}%
  \begin{beamercolorbox}[wd=.15\paperwidth,ht=2.25ex,dp=1ex,center]{date in head/foot}%
    \usebeamerfont{date in head/foot}\insertframenumber{} / \inserttotalframenumber
  \end{beamercolorbox}}%
  \vskip0pt%
}

%==============================================================================
%     TITLE
%==============================================================================
\title[HIL Plant Modeling]{Hardware-in-the-Loop Plant Modeling for Autonomous Vehicle}

\author[H.~Grady, N.~Nauman]{Hannah~Grady \and Nicholas~Nauman 
\linebreak Advisor:~Dr.~Suruz~Miah}
% - Give the names in the same order as the appear in the paper.
% - Use the \inst{?} command only if the authors have different
%   affiliation.

\institute[Bradley University] % (optional, but mostly needed)
{
  Department of Electrical and Computer Engineering\\
  Bradley University\\
  1501 W. Bradley Avenue\\
  Peoria, IL, 61625, USA
}
% - Use the \inst command only if there are several affiliations.
% - Keep it simple, no one is interested in your street address.

\date[April~29,~2022]{Tuesday, April~29,~2022}

% - Either use conference name or its abbreviation.
% - Not really informative to the audience, more for people (including
%   yourself) who are reading the slides online

\logo{\hfill\href{http://www.bradley.edu}{\includegraphics[width=0.75cm]{figs/logoBU1-Print}}}  % place logo in every page 


\subject{Mobile Robot Localization}
% This is only inserted into the PDF information catalog. Can be left
% out. 

% If you have a file called "university-logo-filename.xxx", where xxx
% is a graphic format that can be processed by latex or pdflatex,
% resp., then you can add a logo as follows:

% \pgfdeclareimage[height=0.5cm]{university-logo}{university-logo-filename}
% \logo{\pgfuseimage{university-logo}}

% Delete this, if you do not want the table of contents to pop up at
% the beginning of each subsection:
\AtBeginSection[]
{
  \begin{frame}<beamer>{Outline}
    \tableofcontents[currentsection,currentsubsection]
  \end{frame}
}

%==============================================================================
%==============================================================================
%     START OF SLIDES
%==============================================================================
%==============================================================================

% Let's get started
\begin{document}

\begin{frame}
  \titlepage
\end{frame}

\begin{frame}{Outline} 
  \tableofcontents%[pausesections]
  % You might wish to add the option [pausesections]
\end{frame}

% Section and subsections will appear in the presentation overview
% and table of contents.
\section{Introduction}

\begin{frame}{Introduction}{}

\end{frame}

\begin{frame}{Introduction}{Problem Statement}
  \begin{block}{Problem Statement}
    \begin{LARGE}
      
    \end{LARGE}
  \end{block}
  \pause
  \begin{block}{Proposed Solution}
    \begin{LARGE}
      
    \end{LARGE}
  \end{block}
\end{frame}

%----------------------------------


%----------------------------------

%\begin{frame}{Introduction}{Previous Work}
%  \begin{block}{Existing Solution}
%        Mobile application platform interface with ultrasound and radio transmission technology~\cite{Sales2016-CompaRob}
%  \end{block}
%    % \begin{figure}[b]
%    %     \centering
%    %     \includegraphics[width=0.4\textwidth]{figs/img/CompaRob}
%    %     \caption{CompaRob}
%    %     %\label{fig:sysBlockDiag}
%    % \end{figure}
%\end{frame}
%
%%----------------------------------
%
%\begin{frame}{Introduction}{Previous Work}
%  \begin{block}{Existing Solution}
%        Gated Recurrent Unit (GRU) network with LiDAR sensor and camera to map the customer~\cite{islam_lam_fukuda_kobayashi_kuno_2019}
%  \end{block}
%    % \begin{figure}[b]
%    %     \centering
%    %     \includegraphics[width=0.38\textwidth]{figs/img/ShoppingSuportRobot}
%    %     \caption{Shopping Support Robot}
%    %     %\label{fig:sysBlockDiag}
%    % \end{figure}
%\end{frame}

%----------------------------------
\section{Literature Review}

\begin{frame}{Literature Review}{}

\end{frame}
%---------------------------
\section{System Identification Preliminaries}

\begin{frame}{System Identification Preliminaries}{}

\end{frame}

\begin{frame}{System Identification Preliminaries}{Examples}
  \begin{block}{}
    \begin{LARGE}
     
    \end{LARGE}
  \end{block}
  \begin{block}{}
    \begin{LARGE}
      
    \end{LARGE}
  \end{block}
\end{frame}
%-------------------------
\section{System Architecture}

\begin{frame}{System Architecture}{}

\end{frame}

\begin{frame}{System Architecture}{Subsystems}

\end{frame}

\begin{frame}{System Architecture}{Subsystems}

\end{frame}
%---------------------------
\section{System Modeling using Neural Networks}

\begin{frame}{Modeling using Neural Networks}{}

\end{frame}
%-------------------------
\section{Modeling Vehicle Subsystems}

\begin{frame}{Modeling Vehicle Subsystems}{}

\end{frame}

\begin{frame}{Modeling Vehicle Subsystems}{System Requirements}

\end{frame}
%-------------------------
\section{Validation and Testing}

\begin{frame}{Validation and Testing}{Experimental Setup}

\end{frame}

\begin{frame}{Transfer Function Modeling}{Steering System}

\end{frame}

\begin{frame}{Transfer Function Modeling}{Acceleration System}

\end{frame}

\begin{frame}{Transfer Function Modeling}{Brake System}

\end{frame}

\begin{frame}{Neural Network Modeling}{Steering System}

\end{frame}

\begin{frame}{Neural Network Modeling}{Acceleration System}

\end{frame}

\begin{frame}{Neural Network Modeling}{Brake System}

\end{frame}

%------------------------------------------------------------------------------
%     SECTION BREAK
%------------------------------------------------------------------------------

\section{Conclusions and Future Work}

\begin{frame}{Conclusions and Future Work}
  \begin{block}{Conclusions}
    \begin{itemize}
      \item 
    \end{itemize}
  \end{block}
  \begin{block}{Future Work}
    \begin{itemize}
      \item 
    \end{itemize}
  \end{block}
\end{frame}

%------------------------------------------------------------------------------
%     SECTION BREAK
%------------------------------------------------------------------------------

\section{References}

\begin{frame}{References}
  \bibliographystyle{IEEEtran}
  \bibliography{bib/references.bib}
\end{frame}

%==============================================================================
%==============================================================================
%     END OF SLIDES
%==============================================================================
%==============================================================================

\end{document}

%%% Local Variables:
%%% mode: latex
%%% TeX-master: t
%%% End:
