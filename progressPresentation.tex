% Copyright 2004 by Till Tantau <tantau@users.sourceforge.net>.
%
% In principle, this file can be redistributed and/or modified under
% the terms of the GNU Public License, version 2.
%
% However, this file is supposed to be a template to be modified
% for your own needs. For this reason, if you use this file as a
% template and not specifically distribute it as part of a another
% package/program, I grant the extra permission to freely copy and
% modify this file as you see fit and even to delete this copyright
% notice. 

% \UseRawInputEncoding
\documentclass{beamer}

% There are many different themes available for Beamer. A comprehensive
% list with examples is given here:
% http://deic.uab.es/~iblanes/beamer_gallery/index_by_theme.html
% You can uncomment the themes below if you would like to use a different
% one:
%\usetheme{AnnArbor}
%\usetheme{Antibes}
%\usetheme{Bergen}
%\usetheme{Berkeley}
%\usetheme{Berlin}
%\usetheme{Boadilla}
%\usetheme{boxes}
%\usetheme{CambridgeUS}
%\usetheme{Copenhagen}
%\usetheme{Darmstadt}
%\usetheme{default}
%\usetheme{Frankfurt}
%\usetheme{Goettingen}
%\usetheme{Hannover}
%\usetheme{Ilmenau}
%\usetheme{JuanLesPins}
%\usetheme{Luebeck}
\usetheme{Madrid}
%\usetheme{Malmoe}
%\usetheme{Marburg}
%\usetheme{Montpellier}
%\usetheme{PaloAlto}
%\usetheme{Pittsburgh}
%\usetheme{Rochester}
%\usetheme{Singapore}
%\usetheme{Szeged}
%\usetheme{Warsaw}

\usepackage{pgfgantt}
\usepackage{todonotes}
\usepackage{media9}
\usepackage{fontawesome5}
\usepackage{subfigure}
\usepackage{booktabs,array}
\usepackage{tabulary}
\usepackage{caption}
\usepackage{graphicx}
\usepackage{siunitx}
\usepackage{arydshln}
\usepackage{multicol}

\usepackage[ruled, vlined, linesnumbered]{algorithm2e} % For algorithms

\usepackage{amsmath} % For typesetting math

% Customize Warsaw color 
\setbeamercolor*{palette primary}{use=structure,fg=white,bg=red!50!black}
\setbeamercolor*{palette secondary}{use=structure,fg=white,bg=red!60!black}
\setbeamercolor*{palette tertiary}{use=structure,fg=white,bg=red!70!black}

% Customize Warsaw block title and background colors
\setbeamercolor{block title}{bg=red!50!black,fg=white}

\setbeamertemplate{bibliography item}{\insertbiblabel}  % insert bibliography numbers instead of symbol
\setbeamertemplate{caption}[numbered] % adds the figure or table number to the caption.


\title[HIL Plant Modeling]{Hardware-in-the-Loop Plant Modeling for Autonomous
  Vehicle}

% % A subtitle is optional and this may be deleted
\subtitle{Progress Presentation}

\author[H.~Grady, N.~Nauman]{Hannah~Grady \and Nicholas~Nauman 
\linebreak Advisor:~Dr.~Suruz~Miah}
% - Give the names in the same order as the appear in the paper.
% - Use the \inst{?} command only if the authors have different
%   affiliation.

\institute[Bradley University] % (optional, but mostly needed)
{
  Department of Electrical and Computer Engineering\\
  Bradley University\\
  1501 W. Bradley Avenue\\
  Peoria, IL, 61625, USA
}
% - Use the \inst command only if there are several affiliations.
% - Keep it simple, no one is interested in your street address.

\date[March~1,~2022]{Tuesday, March~1,~2022}

% - Either use conference name or its abbreviation.
% - Not really informative to the audience, more for people (including
%   yourself) who are reading the slides online

\logo{\hfill\href{http://www.bradley.edu}{\includegraphics[width=0.75cm]{figs/logoBU1-Print}}}  % place logo in every page 


% \subject{Mobile Robot Localization}
% This is only inserted into the PDF information catalog. Can be left
% out. 

% If you have a file called "university-logo-filename.xxx", where xxx
% is a graphic format that can be processed by latex or pdflatex,
% resp., then you can add a logo as follows:

% \pgfdeclareimage[height=0.5cm]{university-logo}{university-logo-filename}
% \logo{\pgfuseimage{university-logo}}

% Delete this, if you do not want the table of contents to pop up at
% the beginning of each subsection:
\AtBeginSubsection[]
{
  \begin{frame}<beamer>{Outline}
    \tableofcontents[currentsection,currentsubsection]
  \end{frame}
}

% Delete this, if you do not want the table of contents to pop up at
% the beginning of each section:
\AtBeginSection[]
{
  \begin{frame}<beamer>{Outline}
    \tableofcontents[currentsection]
  \end{frame}
}

% Let's get started
\begin{document}

\begin{frame}
  \titlepage
\end{frame}

\begin{frame}{Outline} 
  \tableofcontents%[pausesections]
  % You might wish to add the option [pausesections]
\end{frame}

% Section and subsections will appear in the presentation overview
% and table of contents.
\section{Introduction}

\begin{frame}{Introduction}{}
    \begin{block}{}
    	\begin{itemize}
    		\item 
		\end{itemize}
    \end{block}
        
\end{frame}


%----------------------------------

\section{Steering Subsystem}

\begin{frame}{Steering Subsystem}
  \begin{block}{}
 \begin{itemize}
        \item 
	\end{itemize}
	
  \end{block}

\end{frame}


%----------------------------------

\section{Acceleration Subsystem}

\begin{frame}{Acceleration Subsystem}
  \begin{block}{}
 \begin{itemize}
        \item 
	\end{itemize}
	
  \end{block}
 
\end{frame}

%--------------------------------

\section{Brake Subsystem}

\begin{frame}{Brake Subsystem}
  \begin{block}{Brake Model}
 \begin{itemize}
        \item System Identification Toolbox
        \item Neural Net Time Series Toolbox
	\end{itemize}
	
  \end{block}

\end{frame}


%----------------------------------


\section{Concluding Remarks}
\begin{frame}{Concluding Remarks}
  \begin{block}{Semester goals}
    %\begin{LARGE}
      \begin{itemize}
        \item 
      \end{itemize}
    %\end{LARGE}
  \end{block}
  \begin{block}{Anticipated Challenges}
    \begin{itemize}
      \item 
    \end{itemize}
  \end{block}
\end{frame}

%----------------------------------

\section{References}

%\begin{frame}{References}
%  \bibliographystyle{IEEEtran}
%   \begin{itemize}
%     \item N. Rawashdeh, R. Haddad, O. Jadallah, and A. To’ma, “A person-following
% robotic cart controlled via a smartphone application: design and evaluation,”
% 09 2017, pp. 1–5.
%     \item M. M. Islam, A. Lam, H. Fukuda, Y. Kobayashi, and Y. Kuno, “An intelligent
% shopping support robot: understanding shopping behavior from 2d skeleton data using gru network,” ROBOMECH Journal, vol. 6, no. 1, 2019.
%     \item J. Sales, J. Marti, R. Marin Prades, E. Cervera, and P. Sanz, “Comparob: The
% shopping cart assistance robot,” International Journal of Distributed Sensor Networks, vol. 2016, pp. 1–15, 02 2016.
%     \item M. S. Miah, J. Knoll, and K. Hevrdejs, “Intelligent range-only mapping and navigation for mobile robots,” IEEE Transactions on Industrial Informatics, vol. 14, no. 3, pp. 1164–1174, 2018.
%     \item D. Li and S. Lane, “A novel and versatile parabolic reflector that significantly
% improves wi-fi reception at different distances and angles,” 2013.
  
%   \end{itemize}


% \end{frame}

%----------------------------------

\begin{frame}[allowframebreaks]{References}

  \bibliographystyle{IEEEtran}
  \bibliography{bib/references.bib}

%\begin{itemize}
%\item
%@INPROCEEDINGS{Donjaroennon2021,
%  author={Donjaroennon, Natthapon and Nuchkum, Suphatchakan and Leeton, Uthen},
%  booktitle={2021 9th International Electrical Engineering Congress (iEECON)}, 
%  title={Mathematical model construction of DC Motor by closed-loop system Identification technique Using Matlab/Simulink}, 
%  year={2021},
%  volume={},
%  number={},
%  pages={289-292},
%  doi={10.1109/iEECON51072.2021.9440305}}
%\end{itemize}
% 	\begin{itemize}
% 		\item T. Xie, H. Jiang, X. Zhao, and C. Zhang, “A wi-fi-based wireless indoor position sensing system with multipath interference mitigation,” Sep 2019. [Online].
% Available: https://www.ncbi.nlm.nih.gov/pmc/articles/PMC6767237/
% 		\item A. M. Ladd, K. E. Bekris, A. Rudys, L. E. Kavraki, and D. S. Wallach, “Robotics-
% based location sensing using wireless ethernet,” Wireless Networks, vol. 11, no. 1-2, p. 189–204, 2005.
% 		\item M. Lindhe, K. Johansson, and A. Bicchi, “An experimental study of exploiting multipath fading for robot communications,” Robotics: Science and Systems III, 2007.
% 		\item M. Lindhe and K. Johansson, “Using robot mobility to exploit multipath fading,”Wireless Communications, IEEE, vol. 16, pp. 30 – 37, 03 2009.
% 	\end{itemize}

\end{frame}


\end{document}



%%% Local Variables:
%%% mode: latex
%%% TeX-master: t
%%% End:
